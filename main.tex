\documentclass[12pt,fleqn,twoside,a4paper]{book}

\usepackage[utf8]{inputenc} %% lettere accentate
\usepackage[italian]{babel}

\RequirePackage{pdf14}
\usepackage[a-1b]{pdfx}

\usepackage[toc,page]{appendix}
\usepackage[T1]{fontenc}
\usepackage[procnames]{listings}
\usepackage{quoting}
\usepackage{csquotes}
\usepackage{xcolor}
\usepackage{eurosym}
\usepackage{ragged2e}
\usepackage{fancyhdr}
\usepackage{caption}
\usepackage{subfiles}
\usepackage{graphicx}

%\usepackage{booktabs}
\graphicspath{{immagini}{../immagini/}}

\setlength{\headheight}{15pt}
\makeatletter
\renewcommand{\chaptermark}[1]{%
  \ifnum\value{chapter}>0
    \markboth{Capitolo \thechapter{}: #1}{}%
  \else
    \markboth{#1}{}%
  \fi}
\makeatother
\fancypagestyle{frontmatter}{%
  \fancyhf{}% Clear header/footer
  \fancyfoot[C]{\thepage}%
}



\fancypagestyle{mainmatter}{%
  \fancyhf{}% Clear header/footer
  \fancyhead[LE,RO]{\itshape \nouppercase \rightmark}
  \fancyhead[LO,RE]{\itshape \nouppercase Capitolo \arabic{chapter}}
  \fancyfoot[C]{\thepage}
}
\usepackage{afterpage}

\newcommand\blankpage{%
    \null
    \thispagestyle{empty}%
    \addtocounter{page}{-1}%
    \newpage}
%\fancyhead[LE,RO]{\itshape \nouppercase \rightmark}
%\fancyhead[LO,RE]{\itshape \nouppercase Capitolo \arabic{chapter}}

\quotingsetup{font=small}
\DeclareFixedFont{\ttb}{T1}{txtt}{bx}{n}{12} % for bold
\DeclareFixedFont{\ttm}{T1}{txtt}{m}{n}{12}  % for normal

\usepackage[top=2.5cm,margin=3cm,bottom=2.5cm]{geometry}
\usepackage{palatino}
\usepackage{enumitem}
\usepackage{float}
\usepackage{tabulary}
\usepackage{longtable}
\usepackage{tikz}
\usepackage{amsmath}
\usepackage{fancyvrb}
%\usepackage[compact]{titlesec}

\newtheorem{definition}{Definizione}[section]
\newcommand{\pij}[1]{\var{p}\ensuremath{_{#1}}}
\newcommand{\m}[1]{\var{m}\ensuremath{_{#1}}}
\newcommand{\ve}[1]{\var{v}\ensuremath{_{#1}}}
\newcommand{\ed}[1]{\var{e}\ensuremath{_{#1}}}
\newcommand{\e}[1]{\var{evid}\ensuremath{_{#1}}}
\newcommand{\cond}[1]{\var{cond}\ensuremath{_{#1}}}
\newcommand{\A}{{\it Attr\/}}
\newcommand{\M}{{\it M\/}}
\newcommand{\bi}{{\it b\/}}
\newcommand{\toc}{{\it ToC\/}}
\newcommand{\var}[1]{{\it #1\/}}
\newcommand{\Rec}[1]{\ensuremath{R_{#1}\/}}
\newcommand{\ER}[1]{\ensuremath{ER_{#1}\/}}
\newcommand{\EP}[1]{\ensuremath{EP_{#1}\/}}
\newcommand{\Eval}[1]{\ensuremath{Eval_{#1}\/}}



% ABILITA BIBLIO
\usepackage[backend=biber, style=numeric, sorting=none, maxbibnames=5]{biblatex}


\setcounter{biburllcpenalty}{7000}
\setcounter{biburlucpenalty}{8000}

\addbibresource{bibliografia.bib}
\input{linguaggi/python.tex}
\input{linguaggi/html.tex}
\definecolor{lightgray}{rgb}{0.95, 0.95, 0.95}
\definecolor{darkgray}{rgb}{0.4, 0.4, 0.4}
\definecolor{purple}{rgb}{0.65, 0.12, 0.82}
\definecolor{editorGray}{rgb}{0.95, 0.95, 0.95}
\definecolor{editorOcher}{rgb}{1, 0.5, 0} % #FF7F00 -> rgb(239, 169, 0)
\definecolor{editorGreen}{rgb}{0, 0.5, 0} % #007C00 -> rgb(0, 124, 0)



\lstdefinelanguage{XML}{
  basicstyle=\ttfamily\footnotesize,
  morestring=[b]",
  moredelim=[s][\bfseries\color{darkblue}]{<}{\ },
  moredelim=[s][\bfseries\color{darkblue}]{</}{>},
  moredelim=[l][\bfseries\color{darkblue}]{/>},
  moredelim=[l][\bfseries\color{darkblue}]{>},
  morecomment=[s]{<?}{?>},
  morecomment=[s]{<!--}{-->},
  commentstyle=\color{DarkOliveGreen},
  stringstyle=\color{blue},
  identifierstyle=\color{red}
}
 
 
\newcommand\xmlstyle{\lstset{
language=XML,
basicstyle=\ttfamily\footnotesize,
breaklines=true,
tabsize=4,
numbers=left,
numberstyle=\ttfamily\footnotesize,
frame=tb,                         % Any extra options here
showstringspaces=false            % 
  % Code design
  identifierstyle=\color{black},
  keywordstyle=\color{blue}\bfseries,
  ndkeywordstyle=\color{DarkOliveGreen}\bfseries,
  stringstyle=\color{black}\ttfamily,
  commentstyle=\color{darkgray}\ttfamily,
  % Code
  alsodigit={.:;},	
  tabsize=2,
  showtabs=false,
  showspaces=false,
  showstringspaces=false,
  extendedchars=true,
  breaklines=true,
  % German umlauts
  literate=%
  {Ö}{{\"O}}1
  {Ä}{{\"A}}1
  {Ü}{{\"U}}1
  {ß}{{\ss}}1
  {ü}{{\"u}}1
  {ä}{{\"a}}1
  {ö}{{\"o}}1
}}
% HTML environment
\lstnewenvironment{xml}[1][]
{
\xmlstyle
\lstset{#1}
}
{}
\newcommand\xmlinline[1]{{\xmlstyle\lstinline!#1!}}

%\addto{\captionsenglish}{\renewcommand*{\appendixname}{Appendici}}

%%%%%%%%%%%%%
% tabelle lunghe
%%%%%%%%%%%%%

\makeatletter
\def\ltabulary{%
\def\endfirsthead{\\}%
\def\endhead{\\}%
\def\endfoot{\\}%
\def\endlastfoot{\\}%
\def\tabulary{%
  \def\TY@final{%
\def\endfirsthead{\LT@end@hd@ft\LT@firsthead}%
\def\endhead{\LT@end@hd@ft\LT@head}%
\def\endfoot{\LT@end@hd@ft\LT@foot}%
\def\endlastfoot{\LT@end@hd@ft\LT@lastfoot}%
\longtable}%
  \let\endTY@final\endlongtable
  \TY@tabular}%
\dimen@\columnwidth
\advance\dimen@-\LTleft
\advance\dimen@-\LTright
\tabulary\dimen@}
\def\endltabulary{\endtabulary}
\makeatother

%%%%%%%%%%%%%%%%%%%%%%%%%



\AtBeginDocument{\addtocontents{toc}{\protect\thispagestyle{plain}}} 
\AtBeginDocument{\addtocontents{lof}{\protect\thispagestyle{plain}}} 
\AtBeginDocument{\addtocontents{lot}{\protect\thispagestyle{plain}}}

\title{TESI}
\author{Patrizio Tufarolo}
\date{April 2017}
\frontmatter
\makeindex


\begin{document}
%%%%%%%%%%%%%%%%%%%%%%%%%%%%%%%%%%%%%%%%%%%%

\pagestyle{frontmatter}
\begin{titlepage}
  \begin{center}
    \includegraphics[height=5.0cm]{immagini/unimi.jpg}
    
    \vspace*{.4cm}
    {\Large 
      %\emph{Corso di Laurea Magistrale in\\[.3cm]
      %  Scienze e Tecnologie dell'Informazione}
      %\emph{Corso di Laurea Magistrale in Sicurezza Informatica}
      %\emph{Corso di Laurea in Informatica}
      \emph{Corso di Laurea Magistrale in Sicurezza Informatica}
      % \emph{Corso di Laurea in\\[.3cm]
      %   Tecnologie per la Società dell'Informazione}
    }
    \vfill
    \begin{LARGE}
      \textbf{Titolo della tesi}
    \end{LARGE}
    
    \vfill
    \begin{minipage}{\linewidth}
      \begin{tabular}{l r}
        \begin{minipage}{.45\linewidth}
          \begin{flushleft}
            {\large
              RELATORE\\[.2cm]
              Nome relatore \\[.6cm]
              CORRELATORE\\[.2cm]
              Nome correlatore
            }

            % \vspace*{1.5cm}
            % {\large
            %   CORRELATORE\\[.3cm]
            %   Prof. Nome e Cognome
            % }
          \end{flushleft}
        \end{minipage}
        &
        \begin{minipage}{.45\linewidth}
          \begin{flushright}
            {\large
              TESI DI LAUREA DI\\[.3cm]
              Patrizio Tufarolo\\[.45cm]
              Matr. 875041
            }
          \end{flushright}
        \end{minipage}
      \end{tabular}
    \end{minipage}
    
    \vfill
    {\large{{Anno Accademico 2016/2017}}}
  \end{center}
\end{titlepage}

%%%%%%%%%%%%%%%%%%%%%%%%%%%%%%%%%%%%%%%%%


\blankpage
\input{sezioni/dedica.tex}
\input{sezioni/prefazione.tex}
\blankpage
\chapter*{Ringraziamenti}

Ringrazio tutti coloro che mi sono stati vicini in questi cinque anni, credendo in me, supportandomi e spronandomi a fare sempre di meglio.

Ringrazio tutti i membri del SESAR Lab, che mi hanno dato la possibilità di apportare un contributo alle attività di ricerca, in particolar modo relative ai progetti CUMULUS e Moon Cloud. Ringrazio in particolare \textbf{Filippo Gaudenzi}, supervisore delle mie attività nel suddetto laboratorio con cui ho maturato un rapporto di amicizia e collaborazione.

Un ulteriore ringraziamento va alla prof. Sara Foresti, che ha accettato l'impegno di essere controrelatore di questo lavoro di tesi.

\begin{flushright}\textit{Patrizio Tufarolo}\end{flushright}

\input{sezioni/indice.tex}
%\input{sezioni/simboli_e_notazioni.tex}
\pagestyle{mainmatter}
\mainmatter
%\documentclass[../main.tex]{subfiles}
\begin{document}
\chapter*{Introduzione}
\addcontentsline{toc}{chapter}{Introduzione}
\markboth{Introduzione}{Introduzione}

Negli ultimi anni, l'adozione del paradigma cloud ha permesso alle organizzazioni di usufruire di vantaggi prestazionali ed economici nell'erogazione dei servizi IT, grazie sia alle caratteristiche di scalabilità ed elasticità proprie dello stesso, che alla possibilità di allocare risorse in modalità \textit{on-demand}.


Tuttavia, la natura distribuita ed automatizzata della cloud introduce numerose problematiche di sicurezza e valutazione del rischio, soprattutto per quelle realtà in procinto di effettuare migrazioni parziali o totali delle loro infrastrutture.

La centralizzazione degli aspetti di sicurezza nelle mani di un \textit{cloud service provider} rappresenta infatti uno dei maggiori limiti dell'approccio, e richiede un rapporto di fiducia reciproca tra il fornitore del servizio e il cliente. Questo rapporto di fiducia, attualmente basato sulla reputazione del provider, rappresenta la base per minimizzare il rischio e limitare il perimetro di attacco, e per stabilire le responsabilità di ogni entità coinvolta nella gestione degli incidenti.


Questo lavoro di tesi si pone all'interno della filiera di ricerca sulla \textit{security assurance}, che mira ad incrementare il livello di attendibilità di un sistema verificandone i comportamenti attesi in caso di fallimento o attacchi.
Una delle tecniche che possono essere utilizzate consiste nella produzione di evidenze fidate e replicabili, basate su attività di testing e monitoraggio.


In particolare, il lavoro di tesi ha come obiettivo la verifica continua e automatica della compliance allo standard FedRAMP, il programma governativo americano per la valutazione del rischio e l'autorizzazione all'utilizzo di servizi cloud nelle agenzie federali, che è stato adottato in diversi domini ad alta criticità come ad esempio Amazon AWS e il Dipartimento della Difesa US.

Ci si è concentrati quindi da un lato sull'analisi e sullo studio dello standard FedRAMP, allo scopo di identificare i controlli di sicurezza di interesse per una valutazione di compliance (descritti nel documento NIST SP 800-53); dall'altro nell'implementazione dei controlli di sicurezza identificati e nella loro integrazione all'interno di Moon Cloud, una piattaforma a micro-servizi per la trasparency, l'assessment e il monitoraggio continuativo di proprietà non funzionali.


Il lavoro di tesi vuole fornire un componente per la valutazione della compliance continua e automatica ai controlli di sicurezza di FedRAMP e assumere un ruolo di supporto per tutti gli attori coinvolti nel processo di autorizzazione, in particolare i fornitori di servizi che vogliano attestarne la \textit{readiness}.

Nell'ambito della tesi è stato inoltre redatto un articolo dal titolo "A security benchmark for OpenStack" che, partendo dal benchmark CIS, identifica alcuni controlli di sicurezza specifici per OpenStack.
Questo è stato sottomesso ed accettato alla conferenza IEEE Cloud 2017 in programma dal 25 al 30 Giugno  ad Honolulu (Hawaii, USA).


L'analisi dei punti chiave di \textit{FedRAMP} ha consentito di individuare alcune metodologie per supportare le attività degli attori coinvolti nel processo di autorizzazione; sono stati quindi implementati due driver per la valutazione automatica e continua dei controlli di sicurezza NIST 800-53.
Al fine di garantire la copertura delle specifiche del programma, sono stati utilizzati due diversi approcci per l'esecuzione dei controlli di sicurezza:
\begin{itemize}
\item \textit{driver per i controlli automatici}, la cui esecuzione avviene in modo autonomo, per l'\textit{assessment} delle proprietà non-funzionali effettivamente implementate nei sistemi informatici;
\item \textit{driver per i controlli ad interazione umana}, effettuati tramite la somministrazione online di questionari, per l'analisi dei processi di business.
\end{itemize}
Questi driver sono stati poi integrati all'interno della piattaforma multi-layer Moon Cloud, e validati attraverso la verifica della compliance della piattaforma Moon Cloud allo standard FedRAMP, tenendo conto dei costi prestazionali dei controlli di sicurezza effettuati.

L'organizzazione dei capitoli è la seguente:
\begin{itemize}
    \item \textbf{Capitolo 1 - Security assurance: lo stato dell'arte e la sfida.}
        \\Nel primo capitolo sono analizzate le problematiche di sicurezza nella migrazione di infrastrutture tradizionali verso la \textit{cloud} illustrando le soluzioni proposte in letteratura.
        Tra queste, vengono approfondite le tecniche di \textit{assurance} basate su certificazione e controllo della \textit{compliance}.
    \item \textbf{Capitolo 2 - Moon Cloud, un framework per il monitoraggio e la security assurance.}
        \\Nel secondo capitolo viene illustrato Moon Cloud, una soluzione software prodotta da alcuni ricercatori del laboratorio SESAR dell'Università degli Studi di Milano, avente come obiettivo la security assurance.
    \item \textbf{Capitolo 3 - FedRAMP - Federal Risk and Authorization Management Program.}
        Nel terzo capitolo viene analizzato FedRAMP, approfondendone i punti chiave, i ruoli e le responsabilità degli attori coinvolti, e identificando
        i possibili approcci per implementarne i controlli di sicurezza all'interno di Moon Cloud, al fine di fornire uno strumento a supporto delle attività di ciascun attore.
    \item \textbf{Capitolo 4 - Implementazione dei controlli di sicurezza FedRAMP in Moon Cloud.}\\
        Nel quarto capitolo viene proposto un approccio ibrido orchestrato tramite l'integrazione di attività di \textit{security assessment} automatiche e ad interazione umana (per i controlli di sicurezza di carattere proceduale e l'analisi dei processi di business)
        e ne viene illustrata una possibile implementazione.
        Nel primo caso, viene presentato il framework OpenSCAP, un ecosistema realizzato da Red Hat per le attività di auditing, e ne viene proposta un'integrazione con Moon Cloud.
        Nel secondo caso, viene presentato un driver Moon Cloud ad interazione umana, che invia questionari personalizzabili tramite template agli utenti interessati, richiedendone la compilazione e creando un report in PDF.
    \item \textbf{Capitolo 5 - Validazione del framework.}\\
        Nel quinto capitolo viene effettuata la validazione del lavoro svolto. Il driver \textit{OpenSCAP} viene eseguito sul deployment di sviluppo dello stesso \textit{Moon Cloud}, per effettuarne un'analisi di sicurezza.
        Come illustrato nel capitolo 2, \textit{Moon Cloud} è organizzato in microservizi gestiti tramite \textit{container}; in particolare, il deployment è effettuato su una IaaS OpenStack e una PaaS Docker.
        L'analisi si articola perciò in tre fasi:
        \begin{enumerate}
            \item valutazione dell'infrastruttura OpenStack sottostante, mediante OpenSCAP e controlli di sicurezza specifici mappati sui requisiti FedRAMP;
            \item valutazione del livello di piattaforma, costituito da un cluster Docker Swarm;
            \item \textit{assessment} dei vari componenti di Moon Cloud e dei canali di comunicazione.
        \end{enumerate}
    \item \textbf{Capitolo 6 - Conclusione e sviluppi futuri}
        Nel capitolo finale vengono mostrati alcuni sviluppi futuri, che spaziano dall'integrazione dei driver sviluppati con altri standard, per arrivare all'utilizzo dell'approccio illustrato in casistiche analoghe al processo di auditing della compliance.
\end{itemize}


%\addcontentsline{toc}{chapter}{Introduzione}
%\chaptermark{Introduzione}
\end{document}

%\clearpage{\pagestyle{plain}\cleardoublepage}
%\input{capitoli/cloud.tex}
%\clearpage{\pagestyle{plain}\cleardoublepage}
%\input{capitoli/testing.tex}
%\clearpage{\pagestyle{plain}\cleardoublepage}
%\input{capitoli/certificazione.tex}
%\clearpage{\pagestyle{plain}\cleardoublepage}
%\input{capitoli/cumulus.tex}
%\clearpage{\pagestyle{plain}\cleardoublepage}
%\input{capitoli/testagent.tex}
%\clearpage{\pagestyle{plain}\cleardoublepage}
%\input{capitoli/openstack.tex}
%\input{capitoli/certifying_openstack.tex}
%\clearpage{\pagestyle{plain}\cleardoublepage}
%\documentclass[../main.tex]{subfiles}
\begin{document}
\chapter{Conclusioni e sviluppi futuri}
\section {Conclusioni}
In questo lavoro di tesi sono state analizzate le problematiche conseguenti alla migrazione di servizi verso la cloud, illustrando le tecniche sviluppate dalla letteratura.
È stato poi presentato \textit{Moon Cloud}, un framework sviluppato dal laboratorio SESAR Lab\footnote{SEcure Service-oriented Architectures Research Lab - \textit{http://sesar.di.unimi.it}} dell'Università degli Studi di Milano, il cui obiettivo è quello di implementare una metodologia automatica per la valutazione della sicurezza, delle performance e di altre proprietà non funzionali, offrendo un processo di assurance basato su attività di auditing e raccolta di evidenze.

Successivamente è stato approfondito \textit{FedRAMP}, il programma federale per l'autorizzazione e l'analisi del rischio per l'adozione dei servizi cloud nelle agenzie governative, caratterizzando le fasi critiche del programma e illustrando come Moon Cloud possa costituire uno strumento di supporto per i vari attori.
Analizzando i controlli di sicurezza di FedRAMP, è stata quindi effettuata una suddivisione tra controlli automatizzabili e controlli procedurali: per i primi è stato proposto un \textit{driver} \textit{Moon Cloud} basato su OpenSCAP, per la seconda tipologia, invece, è stato fornito un driver per l'esecuzione di sondaggi.

Infine, i controlli sviluppati sono stati validati mediante l'analisi della sicurezza del \textit{deployment} multi-tier di un'architettura a microservizi, il framework Moon Cloud stesso, offrendo una prospettiva sulle performance di esecuzione.

Per l'analisi del \textbf{livello infrastrutturale}, è stato utilizzato il driver sviluppato in questo lavoro di tesi, il cui obiettivo è quello di integrare le funzionalità di \textit{OpenSCAP} in \textit{Moon Cloud}, del quale sono state misurate le prestazioni. I risultati ottenuti, ampiamente caratterizzati in un report, hanno evidenziato varie carenze nella configurazione del sistema operativo della macchina fisica, soprattutto dal punto di vista della gestione delle politiche di accesso, la mancanza di meccanismi di \textit{trust} nei \textit{repository} dei pacchetti e l'assenza di meccanismi di auditing.
Per l'analisi del setup \textit{Open Stack}, ci si è basati sul contenuto dell'articolo "A security benchmark for Open Stack" \cite{MyPaper}, del quale è stato effettuato un mapping sui requisiti FedRAMP. I risultati ottenuti evidenziano ancora una volta mancanza di accuratezza nella configurazione degli aspetti di sicurezza. Tuttavia, poiché i sistemi installati sono dedicati all'utilizzo in laboratorio e non orientate alla produzione ed all'erogazione di servizi verso terzi, il livello di sicurezza effettivamente implementato è di qualità accettabile.

Per l'analisi del \textbf{livello di piattaforma}, è stato utilizzato nuovamente il driver \textit{OpenSCAP}, dopodiché è stato effettuato il \textit{benchmarking} della configurazione di sicurezza del cluster \textit{Docker Swarm}. Per ciascuna delle regole non validate, è stato effettuato il mapping sui requisiti FedRAMP violati.
I container Docker vengono eseguiti con utente \textit{root} e senza l'utilizzo di \textit{user namespace}, non sono presenti meccanismi di \textit{mandatory access control} come \textit{AppArmor} o \textit{selinux}. 
 
L'analisi del \textbf{livello applicativo}, infine, si è svolta ispezionando i componenti dell'architettura \textit{Moon Cloud} e i meccanismi di comunicazione tra gli stessi. Sono state rilevate numerose carenze, alcune a livello di configurazione quindi sostenibili in un ambiente di sviluppo (appartenenti alla classe \textit{CM} di FedRAMP), altre a livello implementativo.
I canali di comunicazione tra alcuni componenti (coda per l'esecuzione dei test, connessione al database, comunicazione con il database dei risultati) sono in chiaro; considerando inoltre che le reti \textit{overlay} di \textit{Docker} non sono configurate in modo cifrato (come illustrato in sezione \ref{ref:dockerswarmanalysis}), il rischio di esposizione di informazioni sensibili è alto.

Il database a supporto delle API necessita di livello di sicurezza implementato a causa della tipologia dei dati trattati dalla piattaforma, come ad esempio i dati di autenticazione e le proprietà dei \textit{target} che devono essere verificati.
A tal proposito potrebbe essere integrato un \textit{vault} sicuro, crittografato con meccanismi e algoritmi \textit{FIPS compliant}.

Affinché il progetto Moon Cloud possa ricevere l'autorizzazione provvisoria ad operare dalla JBO FedRAMP, è necessario implementare le soluzioni ai problemi illustrati.
Per quanto riguarda l'aspetto infrastrutturale e di piattaforma, una possibilità è quella di rivolgersi a fornitori di servizi già autorizzati, come ad esempio Azure.
Le carenze del livello applicativo, invece, devono essere gestite in modo opportuno applicando le \textit{remediation} opportune rispetto ai security control NIST 800-53.

\section {Sviluppi futuri}
Questa tesi si presta a numerosi sviluppi futuri, che possono riguardare sia il perfezionamento dei driver Moon Cloud realizzati, sia l'integrazione degli stessi per la valutazione della compliance per altri standard di settore.

La metodologia di \textit{assurance} illustrata, permette sia di effettuare l'assessment automatico di proprietà non funzionali mediante l'esecuzione di test, sia di analizzare proprietà di sicurezza di carattere procedurale, andando ad ispezionare processi di business condotti perlopiù da attori umani.

Pertanto, se da una parte l'approccio fornito può essere utilizzato per effettuare controlli di compliance verso altre normative e standard, (come ad esempio PCI-DSS, HIPAA e la famiglia ISO27000), da un'altra è possibile adattare le tecniche presentate a contesti alternativi, pur utili a garantire livelli di sicurezza elevati all'interno di un'organizzazione.
Uno tra questi, può essere la valutazione delle competenze in ambito di sicurezza informatica dei dipendenti di un'azienda, la valutazione del grado di accettabilità delle misure di sicurezza adottate, l'automazione di valutazioni comportamentali ad esempio con test di \textit{phishing} controllati.


In questo lavoro di tesi si è voluto inoltre dimostrare come la soluzione \textit{Moon Cloud} possa essere adattata a vari casi d'uso fornendo, nello scenario specifico, supporto a ciascuno degli attori coinvolti nel programma FedRAMP.
La capacità di effettuare analisi periodiche e la forte possibilità di integrazione con componenti di terze parti, unite al supporto a tecnologie standard e validate da istituti autorevoli (in questo caso il NIST) permettono a \textit{Moon Cloud} di affermarsi sul mercato come soluzione di monitoraggio per la \textit{security assurance}; inoltre, grazie all'approccio ibrido, fondato sull'integrazione di processi manuali ed automatici, \textit{Moon Cloud} può essere considerato uno strumento valido ed innovativo per la valutazione della compliance.

Il lavoro di analisi svolto su FedRAMP ha, inoltre, l'obiettivo di fornire delle linee guida per tutte quelle realtà, anche operanti fuori dagli Stati Uniti, che vogliano usufruire dei vantaggi della \textit{cloud} nell'erogazione dei propri servizi, pur mantenendo livelli di rischi accettabili. 

Inoltre, i componenti sviluppati possono essere ulteriormente raffinati e perfezionati.
Per quanto riguarda il driver \textit{OpenSCAP}, è possibile produrre \textit{XCCDF} per l'esecuzione di test di sicurezza relativi ad altre tecnologie.
Uno dei limiti del driver è costituito dalla necessità di avere il pacchetto \textit{oscap} installato sul server target; in tal senso, una delle possibili soluzioni adottabili, potrebbe essere quella di effettuare un fork di \textit{OpenSCAP} per implementare l'esecuzione remota dei controlli, riconducendo l'approccio \textit{OpenSCAP} alla metodologia \textit{Moon Cloud}.

La web application fornita per l'erogazione dei questionari nel controllo \textit{Survey}, invece, può essere adattata allo standard \textit{OCIL} mediante la redazione di un foglio \textit{XSLT} per la traduzione da \textit{OCIL} a \textit{JSON Schema}; possono essere inoltre introdotti meccanismi per garantire l'autenticità e l'integrità dei dati inviati, mediante l'utilizzo di dispositivi per la firma digitale e di tecnologie per lo storage sicuro dei report.

Infine, poiché il materiale SCAP relativo al prodotto OpenStack non si è rivelato di qualità idonea per guidare un processo di valutazione della sicurezza, un ulteriore sviluppo potrebbe consistere nell'integrazione del lavoro illustrato nell'articolo \textit{A Security Benchmark for OpenStack}, eventualmente ampliato ed adattato alle soluzioni dei vari partner della OpenStack Foundation.

\end{document}




%%% CAPITOLI VECCHIA TESI %%%
%\subfile{capitoli/introduzione.tex}
%\clearpage{\pagestyle{plain}\cleardoublepage}
%\subfile{capitoli/cloud.tex}
%\clearpage{\pagestyle{plain}\cleardoublepage}
%\subfile{capitoli/testing.tex}
%\clearpage{\pagestyle{plain}\cleardoublepage}
%\subfile{capitoli/certificazione.tex}
%\clearpage{\pagestyle{plain}\cleardoublepage}
%\subfile{capitoli/cumulus.tex}
%\clearpage{\pagestyle{plain}\cleardoublepage}
%\subfile{capitoli/sonde.tex}
%\clearpage{\pagestyle{plain}\cleardoublepage}
%\subfile{capitoli/testagent.tex}
%\clearpage{\pagestyle{plain}\cleardoublepage}
%\subfile{capitoli/openstack.tex}
%\clearpage{\pagestyle{plain}\cleardoublepage}
%\subfile{capitoli/conclusioni.tex}

%%% CAPITOLI MAGISTRALE

\subfile{capitoli/introduzione.tex}
\clearpage{\pagestyle{plain}\cleardoublepage}
\subfile{capitoli/compliance_challenge.tex}
\clearpage{\pagestyle{plain}\cleardoublepage}
\subfile{capitoli/mooncloud_solution.tex}
\clearpage{\pagestyle{plain}\cleardoublepage}
\subfile{capitoli/introduction_to_fedramp.tex}
\clearpage{\pagestyle{plain}\cleardoublepage}
\subfile{capitoli/fedramp_readiness.tex}
\clearpage{\pagestyle{plain}\cleardoublepage}
\subfile{capitoli/security_controls_fedramp.tex}
\clearpage{\pagestyle{plain}\cleardoublepage}
\subfile{capitoli/mooncloud_deployment.tex}
\clearpage{\pagestyle{plain}\cleardoublepage}
\pagestyle{frontmatter}
\appendix
\thispagestyle{plain}


\chapter{Driver di integrazione con OpenSCAP}
\begin{python}
#!/usr/bin/env python
# -*- coding: utf-8 -*-
import sys
import spur
import uuid

from xml.etree import cElementTree as ElementTree

from driver import Driver
from test.ssh_client import CustomSshShell
from azure.storage.blob import BlockBlobService, ContentSettings

__author__ = "Patrizio Tufarolo"
__email__ = "patrizio@tufarolo.eu"

__description__ = "This driver controls a remote OpenSCAP instance through SSH"

ocil_cs = "http://scap.nist.gov/schema/ocil/2"
xccdf_ns = "http://checklists.nist.gov/xccdf/1.1"


class SSHConnection(object):
    def read_ssh_configuration(self, inputs):
        ssh_connection_ti = self.testinstances.get("read_ssh_configuration",
                                                   {})

        assert ssh_connection_ti is not None
        hostname = ssh_connection_ti.get("hostname")
        port = ssh_connection_ti.get("port", 22)
        username = ssh_connection_ti.get("username")
        password = ssh_connection_ti.get("password", None)
        private_key = ssh_connection_ti.get("private_key", None)
        private_key_passphrase = None
        if private_key is not None:
            private_key_passphrase = ssh_connection_ti.get(
                "private_key_passphrase",
                None)

        return hostname, port, username, password, \
            private_key or None, private_key_passphrase or None

    def ssh_connection(self, inputs):
        hostname, port, username, password, \
            private_key, private_key_passphrase = inputs
        self.ssh_client = \
            CustomSshShell(hostname=hostname,
                           username=username,
                           password=password,
                           port=port,
                           private_key=private_key,
                           private_key_passphrase=private_key_passphrase,
                           missing_host_key=spur.ssh.MissingHostKey.warn,
                           shell_type=spur.ssh.ShellTypes.sh
                           )
        return isinstance(self.ssh_client, spur.SshShell)

    def ssh_create_tmp_dir(self, ssh_client_ok):
        self.temp_dir = None
        assert ssh_client_ok and isinstance(self.ssh_client, spur.SshShell)
        self.temp_dir = self.ssh_client.run(
            ["mktemp", "--directory"], encoding="ascii").output.strip()

    def ssh_remove_tmp_dir(self, inputs):
        assert self.temp_dir
        self.ssh_client.run(["rm", "-rf", self.temp_dir])

    def ssh_close(self, inputs):
        self.ssh_client.__exit__()


class XCCDFEvaluator(SSHConnection):
    def read_xccdf_configuration(self, inputs):
        xccdf_ti = self.testinstances.get("read_xccdf_configuration", None)
        xccdf = xccdf_ti.get("xccdf", None)
        profile = xccdf_ti.get("profile", None)
        assert xccdf is not None and profile is not None
        fetch_remote_resources = xccdf_ti.get("fetch_remote_resources", False)
        return xccdf, profile, fetch_remote_resources

    @staticmethod
    def get_xccdf_references(xccdf):
        result = []
        xccdftree = ElementTree.parse(
            "/usr/src/app/test/ssg/{xccdf}-xccdf.xml".format(xccdf=xccdf)
        )
        check_content_refs = xccdftree.findall(
            ".//{%s}check-content-ref" % xccdf_ns
        )

        for check_content_ref in check_content_refs:
            check_content_ref_href_attr = check_content_ref.get("href")
            if check_content_ref_href_attr.startswith("http://") or \
                    check_content_ref_href_attr.startswith("https://"):
                continue
            refhref = check_content_ref.get("href")
            if refhref not in result:
                result.append(refhref)

        return result

    def upload_files(self, inputs):
        xccdf, profile, fetch_remote_resources = inputs
        with self.ssh_client._connect_sftp() as sftp:
            sftp.put(
                "/usr/src/app/test/ssg/{xccdf}-xccdf.xml".format(xccdf=xccdf),
                self.temp_dir + "/" + "{xccdf}-xccdf.xml".format(xccdf=xccdf)
            )

            for document in self.get_xccdf_references(xccdf):
                sftp.put(
                    "/usr/src/app/test/ssg/{document}"
                    .format(document=document),
                    self.temp_dir + "/" + "{document}".
                    format(document=document)
                )
        return xccdf, profile, fetch_remote_resources

    def get_report(self, inputs):
        with self.ssh_client._connect_sftp() as sftp:
            sftp.get(
                "{temp_dir}/report.html".format(temp_dir=self.temp_dir),
                "/tmp/report.html"
            )
        return inputs

    def evaluate_xccdf(self, inputs):
        xccdf, profile, fetch_remote_resources = inputs
        ssh_command = [
            "oscap",
            "xccdf",
            "eval",
            "--profile", "{profile}".format(profile=profile),
            "--results-arf", "{temp_dir}/results.arf.xml".format(
                temp_dir=self.temp_dir
            ),
            "--progress",
            "--report",
            "{temp_dir}/report.html".format(
                temp_dir=self.temp_dir
            ),
            "{temp_dir}/{xccdf}-xccdf.xml".format(
                temp_dir=self.temp_dir, xccdf=xccdf
            )
        ]
        index = 7
        if not self.verifyRoot():
            ssh_command = ["sudo"] + ssh_command
            index += 1
        if fetch_remote_resources:
            ssh_command.insert(index, "--fetch-remote-resources")
        print ssh_command
        out = self.ssh_client.run(ssh_command,
                                  stdout=sys.stderr,
                                  stderr=sys.stderr,
                                  use_pty=True,
                                  allow_error=True).output.strip().split("\n")
        initial_result = "pass"
        for line in out:
            if line.strip() == "":
                continue
            if ":" not in line:
                continue
            if line.startswith("Downloading"):
                continue

            splitted = line.split(":")
            if len(splitted) < 2:
                continue

            try:
                initial_result = initial_result and splitted[1] != "fail"
                self.result.put_value(splitted[0], splitted[1])
            except IndexError:
                continue

        return initial_result


class AzureUploader(object):
    def read_azure_configuration(self, inputs):
        azure_ti = self.testinstances.get("read_azure_configuration", {})
        azure_user = azure_ti.get("user", None)
        azure_access_key = azure_ti.get("access_key", None)
        azure_container_name = azure_ti.get("container_name", None)
        self.azure_user = azure_user
        self.azure_access_key = azure_access_key
        self.azure_container_name = azure_container_name
        return inputs

    def upload_to_azure(self, inputs):
        try:
            azure_user, \
                azure_access_key, \
                azure_container_name = \
                self.azure_user, \
                self.azure_access_key, \
                self.azure_container_name

            if azure_user is None or azure_access_key is None \
                    or azure_container_name is None:
                return inputs
            block_blob_service = BlockBlobService(account_name=azure_user,
                                                  account_key=azure_access_key)

            blob_name = str(uuid.uuid4()).replace("-", "")[:10]
            block_blob_service.create_blob_from_path(
                self.azure_container_name, blob_name,
                "/tmp/report.html",
                content_settings=ContentSettings(content_type="text/html")
            )
            report_url = block_blob_service.make_blob_url(
                self.azure_container_name, blob_name
            )
            self.result.put_value("report", report_url)
        except:
            self.logger.error("Error during report upload")
        return inputs


class OpenSCAP_SSH(Driver, XCCDFEvaluator, AzureUploader):

    def verifyOscapInstalled(self, inputs):
        try:
            assert self.ssh_client.run(["/usr/bin/which", "oscap"])
            return True
        except AssertionError:
            return False

        return inputs

    def verifyRoot(self):
        return self.ssh_client.run([
            "/usr/bin/id", "-u"
        ]).output.strip() == "0"

    def appendAtomics(self):
        self.appendAtomic(self.read_ssh_configuration, lambda rollback: False)
        self.appendAtomic(self.ssh_connection, self.ssh_close)
        self.appendAtomic(self.ssh_create_tmp_dir, self.ssh_remove_tmp_dir)
        self.appendAtomic(self.verifyOscapInstalled, lambda rollback: None)
        self.appendAtomic(self.read_xccdf_configuration, lambda rollback: None)
        self.appendAtomic(self.upload_files, lambda rollback: None)
        self.appendAtomic(self.evaluate_xccdf, lambda rollback: None)
        self.appendAtomic(self.read_azure_configuration, lambda rollback: None)
        self.appendAtomic(self.get_report, lambda rollback: None)
        self.appendAtomic(self.upload_to_azure, lambda rollback: None)
        self.appendAtomic(self.ssh_remove_tmp_dir, lambda rollback: None)
        self.appendAtomic(self.ssh_close, lambda rollback: None)
\end{python}
\label{appendix:driver1}

\chapter{FedRAMP readiness}
\begin{xml}
<?xml version="1.0" encoding="UTF-8"?>
<ocil xmlns:xsi="http://www.w3.org/2001/XMLSchema-instance"
  xsi:schemaLocation="http://www.mitre.org/ocil/1.0 ocil.xsd" xmlns="http://www.mitre.org/ocil/1.0">
  <generator>
    <schema_version>1.0</schema_version>
    <timestamp>2017-05-20T12:06:31</timestamp>
    <author organization="Universita degli Studi di Milano">Patrizio Tufarolo</author>
  </generator>

  <document>
    <title>FedRAMP Readiness</title>
    <description>Questionario per la FedRAMP readiness</description>
  </document>

  <boolean_question_test_action question_ref="ocil:mitre.org:question:1" id="ocil:mitre.org:testaction:1">
    <when_true>
      <result>SUCCESS</result>
    </when_true>

    <when_false>
      <result>FAIL</result>
    </when_false>

  </boolean_question_test_action>

  

  [...]



  <boolean_question id="ocil:mitre.org:question:1" model="MODEL_YES_NO">
    <question_text>L'organizzazione e in grado di affrontare processi elettronici ed eventuali contenziosi?</question_text>
  </boolean_question>

  <boolean_question id="ocil:mitre.org:question:2" model="MODEL_YES_NO">
    <question_text>L'organizzazione e in grado di determinare e descrivere in modo chiaro i confini del sistema?</question_text>
  </boolean_question>
  <boolean_question id="ocil:mitre.org:question:3" model="MODEL_YES_NO">
    <question_text>L'organizzazione ha meccanismi per gestire la shared responsability?</question_text>
  </boolean_question>

  <boolean_question id="ocil:mitre.org:question:4" model="MODEL_YES_NO">
    <question_text>E presente un meccanismo di autenticazione a due fattori per l'accesso via rete tramite account privilegiati?</question_text>
  </boolean_question>

  <boolean_question id="ocil:mitre.org:question:5" model="MODEL_YES_NO">
    <question_text>E presente un meccanismo di autenticazione a due fattori per l'accesso via rete tramite account non privilegiati?</question_text>
  </boolean_question>

  <boolean_question id="ocil:mitre.org:question:6" model="MODEL_YES_NO">
    <question_text>E presente un meccanismo di autenticazione a due fattori per l'accesso locale tramite account privilegiati?</question_text>
  </boolean_question>

  <boolean_question id="ocil:mitre.org:question:7" model="MODEL_YES_NO">
    <question_text>Si ha la possibilit\`a di effettuare analisi del codice per le soluzioni software adottate?</question_text>
  </boolean_question>

  <boolean_question id="ocil:mitre.org:question:8" model="MODEL_YES_NO">
    <question_text>Sono predisposte protezioni di confine per l'isolamento fisico degli asset?</question_text>
  </boolean_question>

  <boolean_question id="ocil:mitre.org:question:8" model="MODEL_YES_NO">
    <question_text>Sono predisposte protezioni di confine per l'isolamento logico degli asset?</question_text>
  </boolean_question>

  <boolean_question id="ocil:mitre.org:question:8" model="MODEL_YES_NO">
    <question_text>Si \`e in grado di rimediare a situazioni di rischio elevato entro 30 giorni?</question_text>
  </boolean_question>

  <boolean_question id="ocil:mitre.org:question:9" model="MODEL_YES_NO">
    <question_text>Viene redatto un inventario dei dispositivi?</question_text>
  </boolean_question>

  <boolean_question id="ocil:mitre.org:question:10" model="MODEL_YES_NO">
    <question_text>Si hanno meccanismi di sicurezza per evitare data leakage?</question_text>
  </boolean_question>

  <boolean_question id="ocil:mitre.org:question:11" model="MODEL_YES_NO">
    <question_text>Si hanno meccanismi di crittografia dei dati in transito sulla rete?</question_text>
  </boolean_question>

</ocil>

\end{xml}
\label{appendix:readiness}

% Abilita biblio
\input{sezioni/bibliografia.tex}
%\input{sezioni/indice_analitico.tex}
%%%%%%%%%%%%%%%%%%%%%%%%%%%%%%%%%%%%%%%%%%%%
\end{document}
