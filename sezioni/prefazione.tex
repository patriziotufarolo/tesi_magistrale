\begin{comment}
\chapter*{Prefazione}
La diffusione del modello cloud per l’erogazione in outsourcing di applicazioni distribuite ha posto nuove sfide nel campo della sicurezza informatica.

è sorta dunque l’esigenza di stabilire meccanismi per costruire un rapporto
di reciproca fiducia e trasparenza tra il fornitore del servizio cloud e gli utenti finali.
Proprio in quest’ottica nasce il progetto FP7 CUMULUS, all’interno del quale si colloca il presente lavoro di tesi, il cui obiettivo è quello di costruire un framework di certificazione per i servizi cloud.
Si vuole quindi realizzare un’architettura, basata sull’utilizzo di sonde conformi allo schema di certificazione CUMULUS, che sia in grado di conciliare le esigenze di scalabilità e il polimorfismo dei web services con le tecniche di certificazione tradizionali (es. Common Criteria).
\end{comment}