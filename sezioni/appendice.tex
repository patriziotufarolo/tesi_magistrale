\appendix
\thispagestyle{plain}


\chapter{Driver di integrazione con OpenSCAP}
\begin{python}
#!/usr/bin/env python
# -*- coding: utf-8 -*-
import sys
import spur
import uuid

from xml.etree import cElementTree as ElementTree

from driver import Driver
from test.ssh_client import CustomSshShell
from azure.storage.blob import BlockBlobService, ContentSettings

__author__ = "Patrizio Tufarolo"
__email__ = "patrizio@tufarolo.eu"

__description__ = "This driver controls a remote OpenSCAP instance through SSH"

ocil_cs = "http://scap.nist.gov/schema/ocil/2"
xccdf_ns = "http://checklists.nist.gov/xccdf/1.1"


class SSHConnection(object):
    def read_ssh_configuration(self, inputs):
        ssh_connection_ti = self.testinstances.get("read_ssh_configuration",
                                                   {})

        assert ssh_connection_ti is not None
        hostname = ssh_connection_ti.get("hostname")
        port = ssh_connection_ti.get("port", 22)
        username = ssh_connection_ti.get("username")
        password = ssh_connection_ti.get("password", None)
        private_key = ssh_connection_ti.get("private_key", None)
        private_key_passphrase = None
        if private_key is not None:
            private_key_passphrase = ssh_connection_ti.get(
                "private_key_passphrase",
                None)

        return hostname, port, username, password, \
            private_key or None, private_key_passphrase or None

    def ssh_connection(self, inputs):
        hostname, port, username, password, \
            private_key, private_key_passphrase = inputs
        self.ssh_client = \
            CustomSshShell(hostname=hostname,
                           username=username,
                           password=password,
                           port=port,
                           private_key=private_key,
                           private_key_passphrase=private_key_passphrase,
                           missing_host_key=spur.ssh.MissingHostKey.warn,
                           shell_type=spur.ssh.ShellTypes.sh
                           )
        return isinstance(self.ssh_client, spur.SshShell)

    def ssh_create_tmp_dir(self, ssh_client_ok):
        self.temp_dir = None
        assert ssh_client_ok and isinstance(self.ssh_client, spur.SshShell)
        self.temp_dir = self.ssh_client.run(
            ["mktemp", "--directory"], encoding="ascii").output.strip()

    def ssh_remove_tmp_dir(self, inputs):
        assert self.temp_dir
        self.ssh_client.run(["rm", "-rf", self.temp_dir])

    def ssh_close(self, inputs):
        self.ssh_client.__exit__()


class XCCDFEvaluator(SSHConnection):
    def read_xccdf_configuration(self, inputs):
        xccdf_ti = self.testinstances.get("read_xccdf_configuration", None)
        xccdf = xccdf_ti.get("xccdf", None)
        profile = xccdf_ti.get("profile", None)
        assert xccdf is not None and profile is not None
        fetch_remote_resources = xccdf_ti.get("fetch_remote_resources", False)
        return xccdf, profile, fetch_remote_resources

    @staticmethod
    def get_xccdf_references(xccdf):
        result = []
        xccdftree = ElementTree.parse(
            "/usr/src/app/test/ssg/{xccdf}-xccdf.xml".format(xccdf=xccdf)
        )
        check_content_refs = xccdftree.findall(
            ".//{%s}check-content-ref" % xccdf_ns
        )

        for check_content_ref in check_content_refs:
            check_content_ref_href_attr = check_content_ref.get("href")
            if check_content_ref_href_attr.startswith("http://") or \
                    check_content_ref_href_attr.startswith("https://"):
                continue
            refhref = check_content_ref.get("href")
            if refhref not in result:
                result.append(refhref)

        return result

    def upload_files(self, inputs):
        xccdf, profile, fetch_remote_resources = inputs
        with self.ssh_client._connect_sftp() as sftp:
            sftp.put(
                "/usr/src/app/test/ssg/{xccdf}-xccdf.xml".format(xccdf=xccdf),
                self.temp_dir + "/" + "{xccdf}-xccdf.xml".format(xccdf=xccdf)
            )

            for document in self.get_xccdf_references(xccdf):
                sftp.put(
                    "/usr/src/app/test/ssg/{document}"
                    .format(document=document),
                    self.temp_dir + "/" + "{document}".
                    format(document=document)
                )
        return xccdf, profile, fetch_remote_resources

    def get_report(self, inputs):
        with self.ssh_client._connect_sftp() as sftp:
            sftp.get(
                "{temp_dir}/report.html".format(temp_dir=self.temp_dir),
                "/tmp/report.html"
            )
        return inputs

    def evaluate_xccdf(self, inputs):
        xccdf, profile, fetch_remote_resources = inputs
        ssh_command = [
            "oscap",
            "xccdf",
            "eval",
            "--profile", "{profile}".format(profile=profile),
            "--results-arf", "{temp_dir}/results.arf.xml".format(
                temp_dir=self.temp_dir
            ),
            "--progress",
            "--report",
            "{temp_dir}/report.html".format(
                temp_dir=self.temp_dir
            ),
            "{temp_dir}/{xccdf}-xccdf.xml".format(
                temp_dir=self.temp_dir, xccdf=xccdf
            )
        ]
        index = 7
        if not self.verifyRoot():
            ssh_command = ["sudo"] + ssh_command
            index += 1
        if fetch_remote_resources:
            ssh_command.insert(index, "--fetch-remote-resources")
        print ssh_command
        out = self.ssh_client.run(ssh_command,
                                  stdout=sys.stderr,
                                  stderr=sys.stderr,
                                  use_pty=True,
                                  allow_error=True).output.strip().split("\n")
        initial_result = "pass"
        for line in out:
            if line.strip() == "":
                continue
            if ":" not in line:
                continue
            if line.startswith("Downloading"):
                continue

            splitted = line.split(":")
            if len(splitted) < 2:
                continue

            try:
                initial_result = initial_result and splitted[1] != "fail"
                self.result.put_value(splitted[0], splitted[1])
            except IndexError:
                continue

        return initial_result


class AzureUploader(object):
    def read_azure_configuration(self, inputs):
        azure_ti = self.testinstances.get("read_azure_configuration", {})
        azure_user = azure_ti.get("user", None)
        azure_access_key = azure_ti.get("access_key", None)
        azure_container_name = azure_ti.get("container_name", None)
        self.azure_user = azure_user
        self.azure_access_key = azure_access_key
        self.azure_container_name = azure_container_name
        return inputs

    def upload_to_azure(self, inputs):
        try:
            azure_user, \
                azure_access_key, \
                azure_container_name = \
                self.azure_user, \
                self.azure_access_key, \
                self.azure_container_name

            if azure_user is None or azure_access_key is None \
                    or azure_container_name is None:
                return inputs
            block_blob_service = BlockBlobService(account_name=azure_user,
                                                  account_key=azure_access_key)

            blob_name = str(uuid.uuid4()).replace("-", "")[:10]
            block_blob_service.create_blob_from_path(
                self.azure_container_name, blob_name,
                "/tmp/report.html",
                content_settings=ContentSettings(content_type="text/html")
            )
            report_url = block_blob_service.make_blob_url(
                self.azure_container_name, blob_name
            )
            self.result.put_value("report", report_url)
        except:
            self.logger.error("Error during report upload")
        return inputs


class OpenSCAP_SSH(Driver, XCCDFEvaluator, AzureUploader):

    def verifyOscapInstalled(self, inputs):
        try:
            assert self.ssh_client.run(["/usr/bin/which", "oscap"])
            return True
        except AssertionError:
            return False

        return inputs

    def verifyRoot(self):
        return self.ssh_client.run([
            "/usr/bin/id", "-u"
        ]).output.strip() == "0"

    def appendAtomics(self):
        self.appendAtomic(self.read_ssh_configuration, lambda rollback: False)
        self.appendAtomic(self.ssh_connection, self.ssh_close)
        self.appendAtomic(self.ssh_create_tmp_dir, self.ssh_remove_tmp_dir)
        self.appendAtomic(self.verifyOscapInstalled, lambda rollback: None)
        self.appendAtomic(self.read_xccdf_configuration, lambda rollback: None)
        self.appendAtomic(self.upload_files, lambda rollback: None)
        self.appendAtomic(self.evaluate_xccdf, lambda rollback: None)
        self.appendAtomic(self.read_azure_configuration, lambda rollback: None)
        self.appendAtomic(self.get_report, lambda rollback: None)
        self.appendAtomic(self.upload_to_azure, lambda rollback: None)
        self.appendAtomic(self.ssh_remove_tmp_dir, lambda rollback: None)
        self.appendAtomic(self.ssh_close, lambda rollback: None)
\end{python}
\label{appendix:driver1}

\chapter{FedRAMP readiness}
\begin{xml}
<?xml version="1.0" encoding="UTF-8"?>
<ocil xmlns:xsi="http://www.w3.org/2001/XMLSchema-instance"
  xsi:schemaLocation="http://www.mitre.org/ocil/1.0 ocil.xsd" xmlns="http://www.mitre.org/ocil/1.0">
  <generator>
    <schema_version>1.0</schema_version>
    <timestamp>2017-05-20T12:06:31</timestamp>
    <author organization="Universita degli Studi di Milano">Patrizio Tufarolo</author>
  </generator>

  <document>
    <title>FedRAMP Readiness</title>
    <description>Questionario per la FedRAMP readiness</description>
  </document>

  <boolean_question_test_action question_ref="ocil:mitre.org:question:1" id="ocil:mitre.org:testaction:1">
    <when_true>
      <result>SUCCESS</result>
    </when_true>

    <when_false>
      <result>FAIL</result>
    </when_false>

  </boolean_question_test_action>

  

  [...]



  <boolean_question id="ocil:mitre.org:question:1" model="MODEL_YES_NO">
    <question_text>L'organizzazione e in grado di affrontare processi elettronici ed eventuali contenziosi?</question_text>
  </boolean_question>

  <boolean_question id="ocil:mitre.org:question:2" model="MODEL_YES_NO">
    <question_text>L'organizzazione e in grado di determinare e descrivere in modo chiaro i confini del sistema?</question_text>
  </boolean_question>
  <boolean_question id="ocil:mitre.org:question:3" model="MODEL_YES_NO">
    <question_text>L'organizzazione ha meccanismi per gestire la shared responsability?</question_text>
  </boolean_question>

  <boolean_question id="ocil:mitre.org:question:4" model="MODEL_YES_NO">
    <question_text>E presente un meccanismo di autenticazione a due fattori per l'accesso via rete tramite account privilegiati?</question_text>
  </boolean_question>

  <boolean_question id="ocil:mitre.org:question:5" model="MODEL_YES_NO">
    <question_text>E presente un meccanismo di autenticazione a due fattori per l'accesso via rete tramite account non privilegiati?</question_text>
  </boolean_question>

  <boolean_question id="ocil:mitre.org:question:6" model="MODEL_YES_NO">
    <question_text>E presente un meccanismo di autenticazione a due fattori per l'accesso locale tramite account privilegiati?</question_text>
  </boolean_question>

  <boolean_question id="ocil:mitre.org:question:7" model="MODEL_YES_NO">
    <question_text>Si ha la possibilit\`a di effettuare analisi del codice per le soluzioni software adottate?</question_text>
  </boolean_question>

  <boolean_question id="ocil:mitre.org:question:8" model="MODEL_YES_NO">
    <question_text>Sono predisposte protezioni di confine per l'isolamento fisico degli asset?</question_text>
  </boolean_question>

  <boolean_question id="ocil:mitre.org:question:8" model="MODEL_YES_NO">
    <question_text>Sono predisposte protezioni di confine per l'isolamento logico degli asset?</question_text>
  </boolean_question>

  <boolean_question id="ocil:mitre.org:question:8" model="MODEL_YES_NO">
    <question_text>Si \`e in grado di rimediare a situazioni di rischio elevato entro 30 giorni?</question_text>
  </boolean_question>

  <boolean_question id="ocil:mitre.org:question:9" model="MODEL_YES_NO">
    <question_text>Viene redatto un inventario dei dispositivi?</question_text>
  </boolean_question>

  <boolean_question id="ocil:mitre.org:question:10" model="MODEL_YES_NO">
    <question_text>Si hanno meccanismi di sicurezza per evitare data leakage?</question_text>
  </boolean_question>

  <boolean_question id="ocil:mitre.org:question:11" model="MODEL_YES_NO">
    <question_text>Si hanno meccanismi di crittografia dei dati in transito sulla rete?</question_text>
  </boolean_question>

</ocil>

\end{xml}
\label{appendix:readiness}
