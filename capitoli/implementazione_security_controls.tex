\documentclass[../main.tex]{subfiles}
\begin{document}
\chapter{Implementazione dei controlli di sicurezza FedRAMP in Moon Cloud}
%\addcontentsline{toc}{chapter}{Introduzione}
%\chaptermark{Introduzione}
\section{Introduzione}
In questo capitolo verrà effettuata un'analisi dei controlli di sicurezza elencati nel capitolo precedente, classificandoli in \textit{controlli automatici} e \textit{controlli procedurali}. Dopodiché verra offerta una possibile implementazione per ciascuna delle due tipologie di controlli, i quali verranno integrati in Moon Cloud. Per i controlli automatici sarà utilizzato Open Scap, uno strumento per il l'auditing realizzato da Red Hat e certificato dal NIST. I controlli procedurali invece, eseguono l'\textit{processi di business} e vanno ad indirizzare tutte quelle proprietà di carattere puramente qualitativo per cui è fondamentale l'interazione umana.
Questi saranno implementati con un \textit{driver Moon Cloud ad interazione umana}, che somministra un questionario online ad un target.

\section{Analisi dei controlli di sicurezza}
Verranno ora proposte alcune tabelle di riepilogo dei controlli di sicurezza di FedRAMP, suddivisi per famiglia, dei quali viene indicata la \textit{baseline} di riferimento e la tipologia (\textit{A}, automatizzabile / \textit{P} procedurale (interazione umana) / \textit{A/P} composizione di approccio automatico e manuale).

I controlli di sicurezza, essendo proposti in modo astratto così da poter coprire il più vasto numero di scenari, vanno poi comunque studiati e implementati sulla base delle caratteristiche e del contesto del sistema target.
\subsection{Access Control}
\makeatletter

\begin{ltabulary}{|p{2cm}|p{8cm}|L|p{1cm}|}
  \toprule
    \hline
    \textbf{ID}     &\textbf{Nome}                                                          & \textbf{Baseline} & \textbf{Tipo}  \\    \hline
  \midrule
  \endhead
  \textbf{AC-1 }	    &		 access control policy and procedures                                   &		 Basso 	    &		 A \\ \hline
  \textbf{AC-2 }	    &		 account management 	                                                &		 Basso 	    &		 A/M \\ \hline
AC-2 (1) 	&		 automated system account management 	                                &		 Moderato 	&		 A/M \\ \hline
AC-2 (2) 	&		 removal of temporary / emergency accounts                        	    &		 Moderato 	&		 A \\ \hline
AC-2 (3) 	&		 disable inactive accounts                                              &		 Moderato 	&		 A/M \\ \hline
AC-2 (4) 	&		 automated audit actions 	                                            &		 Moderato 	&		 A \\ \hline
AC-2 (5) 	&		 inactivity logout 	                                                    &		 Moderato 	&		 A \\ \hline
AC-2 (7) 	&		 role-based schemes 	                                                &		 Moderato 	&		 M \\ \hline
AC-2 (9) 	&		 restrictions on use of shared groups / accounts 	                    &		 Moderato 	&		 A \\ \hline
AC-2 (10) 	&		 shared / group account credential termination 	                        &		 Moderato 	&		 A/M \\ \hline
AC-2 (12) 	&		 account monitoring / atypical usage 	                                &		 Moderato 	&		 M \\ \hline
\textbf{AC-3 }   	&		 access enforcement 	                                                &		 Basso 		&		 A/M \\ \hline
\textbf{AC-4 }	    &		 information flow enforcement 	                                        &		 Moderato 	&		 A/M \\ \hline
AC-4 (21) 	&		 physical / logical separation of information flows                     &		 Moderato 	&		 A/M \\ \hline
\textbf{AC-5 }	    &		 separation of duties 	                                                &		 Moderato 	&		 A/M \\ \hline
\textbf{AC-6 }	    &		 least privilege                                                        &		 Moderato 	&		 A/M \\ \hline
AC-6 (1) 	&		 authorize access to security functions 	                            &		 Moderato 	&		 A \\ \hline
AC-6 (2) 	&		 non-privileged access for nonsecurity functions 	                    &		 Moderato 	&		 A \\ \hline
AC-6 (5) 	&		 privileged accounts 	                                                &		 Moderato 	&		 A \\ \hline
AC-6 (9) 	&		 auditing use of privileged functions 	                                &		 Moderato 	&		 A \\ \hline
AC-6 (10) 	&		 prohibit non-privileged users from execution of privileged functions 	&		 Moderato 	&		 A \\ \hline
\textbf{AC-7 }	    &		 unsuccessful logon attempts 	                                        &		 Basso 		&		 A \\ \hline
\textbf{AC-8 }	    &		 system use notification                                                &		 Basso 		&		 A \\ \hline
\textbf{AC-10} 	    &		 concurrent session control                                             &		 Moderato 	&		 A \\ \hline
\textbf{AC-11} 	    &		 session lock                                                           &		 Moderato 	&		 A \\ \hline
AC-11 (1) 	&		 pattern-hiding displays                                                &		 Moderato 	&		 A \\ \hline
\textbf{AC-12} 	    &		 session termination                                                    &		 Moderato 	&		 A \\ \hline
\textbf{AC-14} 	    &		 "permitted actions without identification or authentication"           &		 Basso 		&		 A/M \\ \hline
\textbf{AC-17} 	    &		 remote access                                                         	&		 Basso 		&		 A/M \\ \hline
AC-17 (1) 	&		 automated monitoring / control                                         &		 Moderato 	&		 M \\ \hline
AC-17 (2) 	&		 protection of confidentiality / integrity using encryption             &		 Moderato 	&		 A/M \\ \hline
AC-17 (3) 	&		 managed access control points                      	                &		 Moderato 	&		 A \\ \hline
AC-17 (4) 	&		 privileged commands / access                       	                &		 Moderato 	&		 A/M \ \hline
AC-17 (9) 	&		 disconnect / disable access 	                                        &		 Moderato 	&		 A/M \\ \hline
\textbf{AC-18} 	    &		 wireless access                                                       	&		 Basso 		&		 A/M \\ \hline
AC-18 (1) 	&		 authentication and encryption                    	&		 Moderato 	&		 A/M \\ \hline
\textbf{AC-19} 	    &		 access control for mobile devices                                  	&		 Basso 		&		 A \\ \hline
AC-19 (5) 	&		 full device / container-based encryption                        	    &		 Moderato 	&		 A \\ \hline
\textbf{AC-20} 	    &		 use of external information systems                                	&		 Basso 		&		 A \\ \hline
AC-20 (1) 	&		 limits on authorized use     	                                        &		 Moderato 	&		 A \\ \hline
AC-20 (2) 	&		 portable storage devices                                           	&		 Moderato 	&		 A \\ \hline
\textbf{AC-21} 	    &		 information sharing                                                	&		 Moderato 	&		 M \\ \hline
    \textbf{AC-22} 	    &		 publicly accessible content                                           	&		 Basso 		&		 A/M \\ \hline
\end{ltabulary}
%AC17 1

Abbiamo un totale di 43 controlli, di cui 22 sono completamente automatizzabili, 4 sono di carattere procedurale e necessitano interazione umana, mentre per 16 di questi è necessario adottare un approccio ibrido.
Pertanto possiamo utilizzare un approccio totalmente automatico solo nel 51.1\%  dei casi.


\subsection{Awareness \& training}
\makeatletter

\begin{ltabulary}{|p{2cm}|p{8cm}|L|p{1cm}|}
    \hline
    \textbf{ID}     &\textbf{Nome}                                                          & \textbf{Baseline} & \textbf{Tipo}  \\    \hline
  \endhead

    \textbf{AT-1}     & security awareness and training policy and procedures                  & Basso & P \\ \hline
    \textbf{AT-2}     & security awareness training                                           & Basso & P \\ \hline
    AT-2 (2)          & insider threat                                                        & Moderato & P \\ \hline
    \textbf{AT-3}     & role-based security training                                          & Basso & P \\ \hline
    \textbf{AT-4}     & security training records                                             & Basso & P \\ \hline
\end{ltabulary}
\makeatother
I controlli della famiglia \textit{awareness and training} sono soltanto 5, e per ciascuno di essi è richiesta intrinsecamente l'interazione umana.
Questa, infatti, misura il grado di consapevolezza dell'utente rispetto alle politiche e alle procedure di sicurezza. 

L'approccio ibrido può essere utilizzato qualora si voglia adottare un sistema automatico per la valutazione delle conoscenze dell'utente oppure l'individuazione di minacce provenienti dall'esterno.


\subsection{Audit and accountability}
\makeatletter

\begin{ltabulary}{|p{2cm}|p{8cm}|L|p{1cm}|}
  \toprule
    \hline
    \textbf{ID}     &\textbf{Nome}                                                          & \textbf{Baseline} & \textbf{Tipo}  \\    \hline
  \midrule
  \endhead

  \textbf{AU-1 }		&			"audit and accountability policy and procedures" 		        &						 Basso 		&						 P \\ \hline
  \textbf{AU-2 }		&			 audit events 		                                            &						 Basso 		&						 P\\ \hline
AU-2 (3) 	&			 reviews and updates 		                                    &						 Medio 		&						 P \\ \hline
\textbf{AU-3 }		&			 content of audit records 		                                &						 Basso 		&						 A \\ \hline
AU-3 (1) 	&			 additional audit information                   	        	&						 Medio 		&						 P \\ \hline
\textbf{AU-4 }		&			 audit storage capacity 	                                	&						 Basso 		&						 A \\ \hline
\textbf{AU-5 }		&			 response to audit processing failures 	                    	&						 Basso 		&						 A \\ \hline
\textbf{AU-6 }		&			 "audit review, analysis, and reporting" 	                	&						 Basso 		&						 P \\ \hline
AU-6 (1) 	&			 "process integration"                                      	&						 Medio 		&						 P \\ \hline
AU-6 (3) 	&			 "correlate audit repositories"                         		&						 Medio 		&						 P \\ \hline
\textbf{AU-7 }		&			 audit reduction and report generation 	                    	&						 Medio 		&						 A \\ \hline
AU-7 (1) 	&			 automatic processing                                   		&						 Medio 		&						 A \\ \hline
\textbf{AU-8 }		&			 time stamps                                            		&						 Basso 		&						 A \\ \hline
AU-8 (1) 	&			 synchronization with authoritative time source         		&						 Medio 		&						 A \\ \hline
\textbf{AU-9 }		&			 protection of audit information                        		&						 Basso 		&						 A \\ \hline
AU-9 (2) 	&			 audit backup on separate physical systems / components 		&						 Medio 		&						 A \\ \hline
AU-9 (4) 	&			 access by subset of privileged users 	                    	&						 Medio 		&						 A/P \\ \hline
\textbf{AU-11} 		&			 audit record retention                                 		&						 Basso 		&						 A \\ \hline
\textbf{AU-12} 		&			 audit generation 	                                        	&						 Basso 		&						 A \\ \hline
\end{ltabulary}
\makeatother
Dei 19 controlli sull'\textit{auditing}, 11 possono essere effettuati in maniera completamente automatica, 1 richiede un approccio ibrido e 7 richiedono l'interazione umana.
Per cui si può affermare che nel 63\% dei casi, è adottabile un approccio automatico.



\subsection{Certification, Accreditation \& Security Assessment}
\makeatletter

\begin{ltabulary}{|p{2cm}|p{8cm}|L|p{1cm}|}
    \hline
    \textbf{ID}     &\textbf{Nome}                                                          & \textbf{Baseline} & \textbf{Tipo}  \\    \hline
  \endhead


\textbf{CA-1} 		& "security assessment and authorization policy and procedures" 		& Basso 		& P \\ \hline
\textbf{CA-2} 		& security assessments 	                                            	& Basso 		& P \\ \hline
CA-2 (1) 	        & independent assessors                                            		& Basso 		& P \\ \hline
CA-2 (2) 	        & specialized assessments 	                                        	& Moderato 		& P \\ \hline
CA-2 (3) 	        & external organizations 	                                        	& Moderato 		& P \\ \hline
\textbf{CA-3} 		& system interconnections 	                                        	& Basso 		& P \\ \hline
CA-3 (3) 	        & unclassified non-national security system connections 	           	& Moderato 		& P \\ \hline
CA-3 (5) 	        & restrictions on external system connections                   		& Moderato 		& P \\ \hline
\textbf{CA-5} 		& plan of action and milestones                                    		& Basso 		& P \\ \hline
\textbf{CA-6} 		& security authorization                                        		& Basso 		& P \\ \hline
\textbf{CA-7} 		& continuous monitoring                                         		& Basso 		& P \\ \hline
CA-7 (1) 	        & independent assessment                                        		& Moderato 		& P \\ \hline
\textbf{CA-8} 		& penetration testing 	                                            	& Moderato 		& P \\ \hline
CA-8 (1) 	        & independent penetration agent or team 	                           	& Moderato 		& P \\ \hline
\textbf{CA-9} 		& internal system connections                                   		& Basso 		& P \\ \hline
\end{ltabulary}
\begin{center}
\captionof{table}{Controlli della classe CA} 
\end{center}
\makeatother

I controlli di questa classe sono tutti di tipo procedurale, e non possono essere automatizzati: si tratta infatti di una serie di dichiarazioni che il responsabile della sicurezza IT del provider deve compilare.


\subsection{Configuration Management}
\makeatletter

\begin{ltabulary}{|p{2cm}|p{8cm}|L|p{1cm}|}
    \hline
    \textbf{ID}     &\textbf{Nome}                                                          & \textbf{Baseline} & \textbf{Tipo}  \\    \hline
  \endhead


CM-1      & "configuration management policy and procedures"                             & Basso    & P   \\ \hline
CM-2      & baseline configuration                                                       & Basso    & P   \\ \hline
CM-2 (1)  & previews and updates                                                          & Moderato & P   \\ \hline
CM-2 (2)  & automation support for accuracy / currency                                   & Moderato & P   \\ \hline
CM-2 (3)  & retention of previous configurations                                         & Moderato & P   \\ \hline
CM-2 (7)  & configure systems, components, or devices for high-risk areas                & Moderato & P   \\ \hline
CM-3      & configuration change control                                                 & Moderato & P   \\ \hline
CM-4      & security impact analysis                                                     & Basso    & P   \\ \hline
CM-5      & access restrictions for change                                               & Moderato & A/P \\ \hline
CM-5 (1)  & automated access enforcement / auditing                                      & Moderato & A   \\ \hline
CM-5 (3)  & signed components                                                            & Moderato & A   \\ \hline
CM-5 (5)  & "access restrictions for change | limit production / operational privileges" & Moderato & A   \\ \hline
CM-6      & configuration settings                                                       & Basso    & A/P \\ \hline
CM-6 (1)  & automated central management / application / verification                    & Moderato & P   \\ \hline
CM-7      & least functionality                                                          & Basso    & A/P \\ \hline
CM-7 (1)  & periodic review                                                              & Moderato & P   \\ \hline
CM-7 (2)  & prevent program execution                                                    & Moderato & A   \\ \hline
CM-7 (5)  & authorized software / whitelisting                                           & Moderato & A   \\ \hline
CM-8      & information system component inventory                                       & Basso    & A/P \\ \hline
CM-8 (1)  & updates during installations / removals                                      & Moderato & A/P \\ \hline
CM-8 (3)  & automated unauthorized component detection                                   & Moderato & A   \\ \hline
CM-8 (5)  & no duplicate accounting of components                                        & Moderato & A   \\ \hline
CM-9      & configuration management plan                                                & Moderato & P   \\ \hline
CM-10     & software usage restrictions                                                  & Basso    & A/P \\ \hline
CM-10 (1) & open source software                                                         & Moderato & P   \\ \hline
CM-11     & user-installed software                                                      & Basso    & A/P \\ \hline
\end{ltabulary}
\begin{center}
\captionof{table}{Controlli della classe CM} 
\end{center}

I controlli appartenenti a questa famiglia rientrano nella tipologia "controllo operativo". Molti di questi sono quindi controlli essenzialmente procedurali, in quanto consistono nel verificare l'esistenza e la conformità delle policy per la gestione dei cambiamenti.
Gran parte delle conseguenze di queste policy, però, contengono elementi di carattere tecnico; per questa ragione, in alcuni casi, l'esecuzione degli stessi può avvenire in modo automatico o semi-automatico.
In particolare i meccanismi automatici possono essere utilizzati in 14 casi su 26, garantendo quindi una copertura del 53\%. Di questi 14 controlli effettuabili in maniera automatica, la metà devono però essere eseguiti con un approccio ibrido.
\makeatother


\subsection{Contingency Planning}
\makeatletter

\begin{ltabulary}{|p{2cm}|p{8cm}|L|p{1cm}|}
    \hline
    \textbf{ID}     &\textbf{Nome}                                                          & \textbf{Baseline} & \textbf{Tipo}  \\    \hline
  \endhead


  \textbf{CP-1 } & "contingency planning policy and procedures"   & Basso    & P   \\ \hline
  \textbf{CP-2 } & contingency plan                               & Basso    & P   \\ \hline
CP-2 (1)         & coordinate with related plans                  & Moderato & P   \\ \hline
CP-2 (2)         & capacity planning                              & Moderato & P   \\ \hline
CP-2 (3)         & resume essential missions / business functions & Moderato & P   \\ \hline
CP-2 (8)         & identify critical assets                       & Moderato & P   \\ \hline
\textbf{CP-3 }   & contingency training                           & Basso    & P   \\ \hline
\textbf{CP-4 }   & contingency plan testing                       & Basso    & P   \\ \hline
CP-4 (1)         & coordinate with related plans                  & Moderato & P   \\ \hline
\textbf{CP-6 }   & alternate storage site                         & Moderato & A/P \\ \hline
CP-6 (1)         & separation from primary site                   & Moderato & P   \\ \hline
CP-6 (3)         & accessibility                                  & Moderato & P   \\ \hline
\textbf{CP-7 }   & alternate processing site                      & Moderato & P   \\ \hline
CP-7 (1)         & separation from primary site                   & Moderato & P   \\ \hline
CP-7 (2)         & accessibility                                  & Moderato & P   \\ \hline
CP-7 (3)         & priority of service                            & Moderato & P   \\ \hline
\textbf{CP-8 }   & telecommunications services                    & Moderato & P   \\ \hline
CP-8 (1)         & priority of service provisions                 & Moderato & P   \\ \hline
CP-8 (2)         & single points of failure                       & Moderato & P   \\ \hline
\textbf{CP-9 }   & information system backup                      & Basso    & A/P \\ \hline
CP-9 (1)         & testing for reliability / integrity            & Moderato & A   \\ \hline
CP-9 (3)         & separate storage for critical information      & Moderato & A/P \\ \hline
\textbf{CP-10}   & information system recovery and reconstitution & Basso    & P   \\ \hline
CP-10 (2)        & transaction recovery                           & Moderato & A/P \\ \hline
\end{ltabulary}
\captionof{table}{Controlli della classe CP} 

La realizzazione del piano di contingenza è di carattere prettamente procedurale per definizione; l'obiettivo è infatti quello di studiare gli impatti sul business di possibili incidenti informatici, pianificare le operazioni di \textit{incident response}, un piano per il recupero dall'incidente (\textit{disastery recovery}) e le eventuali misure prese per garantire la \textit{business continuity}.


Proprio le ultime citate sono l'oggetto dei pochi controlli parzialmente automatizzabili: nel caso del \textit{CP-6}, ad esempio, che prevede l'esistenza di una locazione alternativa per lo \textit{storage} dei dati, si può verificare che entrambe le locazioni siano sempre sincronizzate e che le proprietà di confidenzialità e integrità nella ridondanza geografica siano rispettate.


La stessa considerazione può essere fatta per i sistemi di backup dei dati (CP-9) e dei log delle transazioni (CP-10), tenendo anche conto del fatto che le informazioni critiche devono essere trattate separatamente (CP-9 (3)). Eventuali sistemi automatici possono essere utilizzati anche per la verifica dell'affidabilità, l'integrità, il testing (e la effettiva possibilità di recupero dati).

L'approccio automatico può essere usato nel 16\% dei casi.
\makeatother


\subsection{Identification and Authentication}
\makeatletter

\begin{ltabulary}{|p{2cm}|p{8cm}|L|p{1cm}|}
    \hline
    \textbf{ID}     &\textbf{Nome}                                                          & \textbf{Baseline} & \textbf{Tipo}  \\    \hline
  \endhead

\textbf{IA-1} & "identification and authentication policy and procedures"                         & Basso    & P   \\ \hline
\textbf{IA-2} & "identification and authentication (organizational users)"                        & Basso    & P   \\ \hline
IA-2 (1)      & network access to privileged accounts                                             & Basso    & A   \\ \hline
IA-2 (2)      & network access to non-privileged accounts                                         & Moderato & A   \\ \hline
IA-2 (3)      & local access to privileged accounts                                               & Moderato & A   \\ \hline
IA-2 (5)      & "identification and authentication (organizational users) | group authentication" & Moderato & A   \\ \hline
IA-2 (8)      & network access to privileged accounts - replay resistant                          & Moderato & A   \\ \hline
IA-2 (11)     & remote access - separate device                                                   & Moderato & A   \\ \hline
IA-2 (12)     & acceptance of piv credentials                                                     & Basso    & A   \\ \hline
\textbf{IA-3} & device identification and authentication                                          & Moderato & A   \\ \hline
\textbf{IA-4} & identifier management                                                             & Basso    & P   \\ \hline
IA-4 (4)      & identify user status                                                              & Moderato & P   \\ \hline
\textbf{IA-5} & authenticator management                                                          & Basso    & A/P \\ \hline
IA-5 (1)      & password-based authentication                                                     & Basso    & A   \\ \hline
IA-5 (2)      & pki-based authentication                                                          & Moderato & A   \\ \hline
IA-5 (3)      & in-person or trusted third-party registration                                     & Moderato & A   \\ \hline
IA-5 (4)      & automated support for password strength determination                             & Moderato & A   \\ \hline
IA-5 (6)      & protection of authenticators                                                      & Moderato & A   \\ \hline
IA-5 (7)      & no embedded unencrypted static authenticators                                     & Moderato & A   \\ \hline
IA-5 (11)     & hardware token-based authentication                                               & Basso    & A   \\ \hline
\textbf{IA-6} & authenticator feedback                                                            & Basso    & A   \\ \hline
\textbf{IA-7} & cryptographic module authentication                                               & Basso    & A   \\ \hline
\textbf{IA-8} & identification and authentication (non- organizational users)                     & Basso    & A   \\ \hline
IA-8 (1)      & acceptance of piv credentials from other agencies                                 & Basso    & A   \\ \hline
IA-8 (2)      & acceptance of third-party credentials                                             & Basso    & A   \\ \hline
IA-8 (3)      & use of ficam-approved products                                                    & Basso    & A   \\ \hline
IA-8 (4)      & use of ficam-issued profiles                                                      & Basso    & A   \\ \hline
\end{ltabulary}
\captionof{table}{Controlli della classe IA} 

I controlli della famiglia di \textit{identificazione e autenticazione} sono perlopiù di carattere tecnico e pertanto in gran parte automatizzabili (nell'85\% dei casi).
Questi hanno una stretta correlazione con i controlli della famiglia \textit{controllo degli accessi} e di \textit{auditing}, già trattate, e manifestano la necessità per il provider di mantenere \textit{accountability} e \textit{non-repudiation}. Pertanto si tratta di controlli effettuati sulla resistenza delle credenziali agli attacchi, sulla sicurezza del processo di autenticazione e sulla possibilità di ricondurre con esattezza tutte le operazioni svolte sul sistema all'attore che le ha compiute, sia esso un servizio automatico o un utente umano.
Molti di questi, tuttavia, sono anche di carattere ibrido o procedurale, poiché basati sulle dichiarazioni del provider stesso.

\makeatother


\subsection{Incident response}
\makeatletter

\begin{ltabulary}{|p{2cm}|p{8cm}|L|p{1cm}|}
  \toprule
    \hline
    \textbf{ID}     &\textbf{Nome}                                                          & \textbf{Baseline} & \textbf{Tipo}  \\    \hline
  \midrule
  \endhead

\textbf{IR-1} & incident response policy and procedures                      & Basso    & P \\ \hline
\textbf{IR-2} & incident response training                                   & Basso    & P \\ \hline
\textbf{IR-3} & incident response testing                                    & Moderato & P \\ \hline
IR-3 (2)      & coordination with related plans                              & Moderato & P \\ \hline
\textbf{IR-4} & incident handling                                            & Basso    & P \\ \hline
IR-4 (1)      & automated incident handling processes                        & Moderato & P \\ \hline
\textbf{IR-5} & incident monitoring                                          & Basso    & P \\ \hline
\textbf{IR-6} & incident reporting                                           & Basso    & P \\ \hline
IR-6 (1)      & automated reporting                                          & Moderato & P \\ \hline
\textbf{IR-7} & incident response assistance                                 & Basso    & P \\ \hline
IR-7 (1)      & automation support for availability of information / support & Moderato & P \\ \hline
IR-7 (2)      & coordination with external providers                         & Moderato & P \\ \hline
\textbf{IR-8} & incident response plan                                       & Basso    & P \\ \hline
\textbf{IR-9} & information spillage response                                & Moderato & P \\ \hline
IR-9 (1)      & responsible personnel                                        & Moderato & P \\ \hline
IR-9 (2)      & training                                                     & Moderato & P \\ \hline
IR-9 (3)      & post-spill operations                                        & Moderato & P \\ \hline
IR-9 (4)      & exposure to unauthorized personnel                           & Moderato & P \\ \hline
\end{ltabulary}

\makeatother


\subsection{Maintenance}
\makeatletter

\begin{ltabulary}{|p{2cm}|p{8cm}|L|p{1cm}|}
  \toprule
    \hline
    \textbf{ID} & \textbf{Nome}                            & \textbf{Baseline} & \textbf{Tipo} \\ \hline
  \midrule
  \endhead
\textbf{MA-1}   & system maintenance policy and procedures & Basso             & P             \\ \hline
\textbf{MA-2}   & controlled maintenance                   & Basso             & P             \\ \hline
\textbf{MA-3}   & maintenance tools                        & Moderato          & A/P           \\ \hline
MA-3 (1)        & inspect tools                            & Moderato          & A             \\ \hline
MA-3 (2)        & inspect media                            & Moderato          & A             \\ \hline
MA-3 (3)        & prevent unauthorized removal             & Moderato          & A             \\ \hline
\textbf{MA-4}   & nonlocal maintenance                     & Basso             & P             \\ \hline
MA-4 (2)        & document nonlocal maintenance            & Moderato          & P             \\ \hline
\textbf{MA-5}   & maintenance personnel                    & Basso             & P             \\ \hline
MA-5 (1)        & individuals without appropriate access   & Moderato          & P             \\ \hline
\textbf{MA-6}   & timely maintenance                       & Moderato          & A/P           \\ \hline
\end{ltabulary}
\makeatother


\subsection{Media protection}
\makeatletter

\begin{ltabulary}{|p{2cm}|p{8cm}|L|p{1cm}|}
  \toprule
    \hline
    \textbf{ID}     &\textbf{Nome}                                                          & \textbf{Baseline} & \textbf{Tipo}  \\    \hline
  \midrule
  \endhead
\textbf{MP-1} & media protection policy and procedures & Basso    & P \\ \hline
\textbf{MP-2} & media access                           & Basso    & P \\ \hline
\textbf{MP-3} & media marking                          & Moderato & P \\ \hline
\textbf{MP-4} & media storage                          & Moderato & P \\ \hline
\textbf{MP-5} & media transport                        & Moderato & A/P \\ \hline
MP-5 (4)      & cryptographic protection               & Moderato & A/P \\ \hline
\textbf{MP-6} & media sanitization                     & Basso    & P \\ \hline
MP-6 (2)      & equipment testing                      & Moderato & P \\ \hline
\textbf{MP-7} & media use                              & Basso    & A/P \\ \hline
MP-7 (1)      & prohibit use without owner             & Moderato & A/P \\ \hline
\end{ltabulary}
\makeatother


\subsection{Physical and environmental protection}
\makeatletter

\begin{ltabulary}{|p{2cm}|p{8cm}|L|p{1cm}|}
    \hline
    \textbf{ID} & \textbf{Nome}                                                 & \textbf{Baseline} & \textbf{Tipo} \\ \hline
  \endhead
\textbf{PE-1 }  & "physical and environmental protection policy and procedures" & Basso             & A/P           \\ \hline
\textbf{PE-2 }  & physical access authorizations                                & Basso             & P             \\ \hline
\textbf{PE-3 }  & physical access control                                       & Basso             & A/P           \\ \hline
\textbf{PE-4 }  & access control for transmission medium                        & Moderato          & A/P           \\ \hline
\textbf{PE-5 }  & access control for output devices                             & Moderato          & A/P           \\ \hline
\textbf{PE-6 }  & monitoring physical access                                    & Basso             & A/P           \\ \hline
PE-6 (1)        & intrusion alarms / surveillance equipment                     & Moderato          & A/P           \\ \hline
\textbf{PE-8 }  & visitor access records                                        & Basso             & A/P           \\ \hline
\textbf{PE-9 }  & power equipment and cabling                                   & Moderato          & A/P           \\ \hline
\textbf{PE-10}  & emergency shutoff                                             & Moderato          & A/P           \\ \hline
\textbf{PE-11}  & emergency power                                               & Moderato          & A/P           \\ \hline
\textbf{PE-12}  & emergency lighting                                            & Basso             & A/P           \\ \hline
\textbf{PE-13}  & fire protection                                               & Basso             & A/P           \\ \hline
PE-13 (2)       & suppression devices / systems                                 & Moderato          & A/P           \\ \hline
PE-13 (3)       & automatic fire suppression                                    & Moderato          & A/P           \\ \hline
\textbf{PE-14}  & temperature and humidity controls                             & Basso             & A/P           \\ \hline
PE-14 (2)       & monitoring with alarms / notifications                        & Moderato          & A/P           \\ \hline
\textbf{PE-15}  & water damage protection                                       & Basso             & A/P           \\ \hline
\textbf{PE-16}  & delivery and removal                                          & Basso             & A/P           \\ \hline
\textbf{PE-17}  & alternate work site                                           & Moderato          & A/P           \\ \hline
\end{ltabulary}
\begin{center}
\captionof{table}{Controlli della classe PE} 
\end{center}

Questa categoria di controlli riguarda la sicurezza dell'ambiente fisico nel quale operano i sistemi del fornitore di servizi.
Si tratta di aspetti in gran parte controllabili tramite metodologie automatiche e manuali: l'eventuale presenza di certificazioni per gli impianti con controlli a cadenza periodica può già essere un'ottima metrica di valutazione; eventuali meccanismi di monitoraggio automatico possono essere adottati mediante l'integrazione con eventuali dispositivi e sonde IoT.


\makeatother


\subsection{Planning}
\begin{ltabulary}{|p{2cm}|p{8cm}|L|p{1cm}|}
    \hline
    \textbf{ID}     &\textbf{Nome}                                                          & \textbf{Baseline} & \textbf{Tipo}  \\    \hline
  \endhead
\textbf{PL-1} & security planning policy and procedures              & Basso    & P \\ \hline
\textbf{PL-2} & system security plan                                 & Basso    & P \\ \hline
PL-2 (3)      & plan / coordinate with other organizational entities & Moderato & P \\ \hline
\textbf{PL-4} & rules of behavior                                    & Basso    & P \\ \hline
PL-4 (1)      & social media and networking restrictions             & Moderato & P \\ \hline
\textbf{PL-8} & information security architecture                    & Moderato & P \\ \hline
\end{ltabulary}
\begin{center}
\captionof{table}{Controlli della classe PL} 
\end{center}

I controlli appartenenti a questa categoria riguardano la redazione del \textit{system security plan} già trattato nel capitolo precedente. Si tratta, ancora una volta, di una serie di informazioni che il responsabile della sicurezza IT deve fornire.


\subsection{Personnel security}
\begin{ltabulary}{|p{2cm}|p{8cm}|L|p{1cm}|}
    \hline
    \textbf{ID}     &\textbf{Nome}                                                          & \textbf{Baseline} & \textbf{Tipo}  \\    \hline
  \endhead
\textbf{PS-1} & personnel security policy and procedures     & Basso    & P \\ \hline
\textbf{PS-2} & position risk designation                    & Basso    & P \\ \hline
\textbf{PS-3} & personnel screening                          & Basso    & P \\ \hline
PS-3 (3)      & information with special protection measures & Moderato & P \\ \hline
\textbf{PS-4} & personnel termination                        & Basso    & P \\ \hline
\textbf{PS-5} & personnel transfer                           & Basso    & P \\ \hline
\textbf{PS-6} & access agreements                            & Basso    & P \\ \hline
\textbf{PS-7} & third-party personnel security               & Basso    & P \\ \hline
\textbf{PS-8} & personnel sanctions                          & Basso    & P \\ \hline
\end{ltabulary}
\captionof{table}{Controlli della classe PS} 
Questa classe di controlli è riferita alla sicurezza e al controllo del comportamento del personale IT. Si tratta di controlli di carattere procedurale.
Le informazioni relativa ai controlli di \textit{Personnel Security} possono essere rilasciate dal responsabile della sicurezza interno al provider e possono
essere eventualmente comprovate mediante la somministrazione periodica di questionari al personale stesso.


\subsection{Risk assessment}
\makeatletter

\begin{ltabulary}{|p{2cm}|p{8cm}|L|p{1cm}|}
  \toprule
    \hline
    \textbf{ID} & \textbf{Nome}                                             & \textbf{Baseline} & \textbf{Tipo} \\ \hline
  \midrule
  \endhead
\textbf{RA-1}   & risk assessment policy and procedures                     & Basso             & P             \\ \hline
\textbf{RA-2}   & security categorization                                   & Basso             & P             \\ \hline
\textbf{RA-3}   & risk assessment                                           & Basso             & P             \\ \hline
\textbf{RA-5}   & vulnerability scanning                                    & Basso             & A/P           \\ \hline
RA-5 (1)        & update tool capability                                    & Moderato          & A             \\ \hline
RA-5 (2)        & update by frequency / prior to new scan / when identified & Moderato          & A             \\ \hline
RA-5 (3)        & breadth / depth of coverage                               & Moderato          & A/P           \\ \hline
RA-5 (5)        & privileged access                                         & Moderato          & A             \\ \hline
RA-5 (6)        & automated trend analyses                                  & Moderato          & A             \\ \hline
RA-5 (8)        & review historic audit logs                                & Moderato          & P             \\ \hline
\end{ltabulary}
\makeatother


\subsection{System and services acquisition}
\makeatletter

\begin{ltabulary}{|p{2cm}|p{8cm}|L|p{1cm}|}
  \toprule
    \hline
    \textbf{ID} & \textbf{Nome}                                              & \textbf{Baseline} & \textbf{Tipo} \\ \hline
  \midrule
  \endhead
\textbf{SA-1 }  & "system and services acquisition policy and procedures"    & Basso             & P             \\ \hline
\textbf{SA-2 }  & allocation of resources                                    & Basso             & P             \\ \hline
\textbf{SA-3 }  & system development life cycle                              & Basso             & P             \\ \hline
\textbf{SA-4 }  & acquisition process                                        & Basso             & P             \\ \hline
SA-4 (1)        & functional properties of security controls                 & Moderato          & P             \\ \hline
SA-4 (2)        & design / implementation information for security controls  & Moderato          & P             \\ \hline
SA-4 (8)        & continuous monitoring plan                                 & Moderato          & P             \\ \hline
SA-4 (9)        & functions / ports / protocols / services in use            & Moderato          & P             \\ \hline
SA-4 (10)       & use of approved piv products                               & Moderato          & P             \\ \hline
\textbf{SA-5 }  & information system documentation                           & Basso             & P             \\ \hline
\textbf{SA-8 }  & security engineering principles                            & Moderato          & P             \\ \hline
\textbf{SA-9 }  & external information system services                       & Basso             & P             \\ \hline
SA-9 (1)        & "risk assessments / organizational approvals"              & Moderato          & P             \\ \hline
SA-9 (2)        & identification of functions / ports / protocols / services & Moderato          & P             \\ \hline
SA-9 (4)        & consistent interests of consumers and providers            & Moderato          & P             \\ \hline
SA-9 (5)        & processing, storage, and service location"                 & Moderato          & P             \\ \hline
\textbf{SA-10}  & developer configuration management                         & Moderato          & P             \\ \hline
SA-10 (1)       & software / firmware integrity verification"                & Moderato          & P             \\ \hline
\textbf{SA-11}  & developer security testing and evaluation                  & Moderato          & P             \\ \hline
SA-11 (1)       & static code analysis                                       & Moderato          & P             \\ \hline
SA-11 (2)       & threat and vulnerability analyses                          & Moderato          & P             \\ \hline
SA-11 (8)       & dynamic code analysis                                      & Moderato          & P             \\ \hline
\end{ltabulary}
\makeatother


\subsection{System and communication protection}
\makeatletter

\begin{ltabulary}{|p{2cm}|p{8cm}|L|p{1cm}|}
    \hline
    \textbf{ID} & \textbf{Nome}                                                             & \textbf{Baseline} & \textbf{Tipo} \\ \hline
  \endhead
\textbf{SC-1 }  & "system and communications protection policy and procedures"              & Basso             & P             \\ \hline
\textbf{SC-2 }  & application partitioning                                                  & Moderato          & A             \\ \hline
\textbf{SC-4 }  & information in shared resources                                           & Moderato          & A             \\ \hline
\textbf{SC-5 }  & denial of service protection                                              & Basso             & A             \\ \hline
\textbf{SC-6 }  & resource availability                                                     & Moderato          & A             \\ \hline
\textbf{SC-7 }  & boundary protection                                                       & Basso             & A             \\ \hline
SC-7 (3)        & access points                                                             & Moderato          & A             \\ \hline
SC-7 (4)        & external telecommunications services                                      & Moderato          & A             \\ \hline
SC-7 (5)        & deny by default / allow by exception                                      & Moderato          & A             \\ \hline
SC-7 (7)        & prevent split tunneling for remote devices                                & Moderato          & A             \\ \hline
SC-7 (8)        & route traffic to authenticated proxy servers                              & Moderato          & A             \\ \hline
SC-7 (12)       & host-based protection                                                     & Moderato          & A             \\ \hline
SC-7 (13)       & isolation of security tools / mechanisms / support components"            & Moderato          & A             \\ \hline
SC-7 (18)       & fail secure                                                               & Moderato          & A             \\ \hline
\textbf{SC-8 }  & transmission confidentiality and integrity                                & Moderato          & A             \\ \hline
SC-8 (1)        & cryptographic or alternate physical protection                            & Moderato          & A             \\ \hline
\textbf{SC-10}  & network disconnect                                                        & Moderato          & A             \\ \hline
\textbf{SC-12}  & cryptographic key establishment and management"                           & Basso             & A             \\ \hline
SC-12 (2)       & symmetric keys                                                            & Moderato          & A             \\ \hline
SC-12 (3)       & asymmetric keys                                                           & Moderato          & A             \\ \hline
\textbf{SC-13}  & cryptographic protection                                                  & Basso             & A             \\ \hline
\textbf{SC-15}  & collaborative computing devices                                           & Basso             & A             \\ \hline
\textbf{SC-17}  & public key infrastructure certificates                                    & Moderato          & A             \\ \hline
\textbf{SC-18}  & mobile code                                                               & Moderato          & A             \\ \hline
\textbf{SC-19}  & voice over internet protocol                                              & Moderato          & A             \\ \hline
\textbf{SC-20}  & "secure name /address resolution service (authoritative source)"          & Basso             & A             \\ \hline
\textbf{SC-21}  & "secure name /address resolution service (recursive or caching resolver)" & Basso             & A             \\ \hline
\textbf{SC-22}  & "architecture and provisioning for name/address resolution service"       & Basso             & A             \\ \hline
\textbf{SC-23}  & session authenticity                                                      & Moderato          & A             \\ \hline
\textbf{SC-28}  & protection of information at rest                                         & Moderato          & A             \\ \hline
SC-28 (1)       & cryptographic protection                                                  & Moderato          & A             \\ \hline
\textbf{SC-39}  & process isolation                                                         & Basso             & A             \\ \hline
\end{ltabulary}
\captionof{table}{Controlli della classe SC} 
I controlli di "System and communication protection policy and procedures" mirano a verificare l'applicazione delle politiche su sistemi e meccanismi di comunicazione. hanno tutti un immediato risvolto tecnico, per questa ragione sono totalmente automatizzabili, con l'eccezione di \textit{SC-1} che ne definisce la parte procedurale.





\subsection{System and information integrity}
\begin{ltabulary}{|p{2cm}|p{8cm}|L|p{1cm}|}
    \hline
    \textbf{ID} & \textbf{Nome}                                               & \textbf{Baseline} & \textbf{Tipo} \\ \hline
  \endhead
\textbf{SI-1}   & system and information integrity policy and procedures      & Basso             & P             \\ \hline
\textbf{SI-2}   & flaw remediation                                            & Basso             & A/P           \\ \hline
SI-2 (2)        & automated flaw remediation status                           & Moderato          & A/P           \\ \hline
SI-2 (3)        & time to remediate flaws / benchmarks for corrective actions & Moderato          & A/P           \\ \hline
\textbf{SI-3}   & malicious code protection                                   & Basso             & A             \\ \hline
SI-3 (1)        & central management                                          & Moderato          & A             \\ \hline
SI-3 (2)        & automatic updates                                           & Moderato          & A             \\ \hline
SI-3 (7)        & nonsignature-based detection                                & Moderato          & A             \\ \hline
\textbf{SI-4}   & information system monitoring                               & Basso             & A             \\ \hline
SI-4 (14)       & wireless intrusion detection                                & Moderato          & A             \\ \hline
SI-4 (16)       & correlate monitoring information                            & Moderato          & A             \\ \hline
SI-4 (1)        & system-wide intrusion detection system                      & Moderato          & A             \\ \hline
SI-4 (23)       & host-based devices                                          & Moderato          & A             \\ \hline
SI-4 (2)        & automated tools for real-time analysis                      & Moderato          & A             \\ \hline
SI-4 (4)        & inbound and outbound communications traffic                 & Moderato          & A             \\ \hline
SI-4 (5)        & system-generated alerts                                     & Moderato          & A             \\ \hline
\textbf{SI-5}   & security alerts, advisories, and directives                 & Basso             & A             \\ \hline
\textbf{SI-6}   & security function verification                              & Moderato          & A             \\ \hline
\textbf{SI-7}   & software, firmware, and information integrity               & Moderato          & A             \\ \hline
SI-7 (1)        & integrity checks                                            & Moderato          & A             \\ \hline
SI-7 (7)        & integration of detection and response                       & Moderato          & A             \\ \hline
\textbf{SI-8}   & spam protection                                             & Moderato          & A             \\ \hline
SI-8 (1)        & central management                                          & Moderato          & A             \\ \hline
SI-8 (2)        & automatic updates                                           & Moderato          & A             \\ \hline
\textbf{SI-10}  & information input validation                                & Moderato          & A             \\ \hline
\textbf{SI-11}  & error handling                                              & Moderato          & A             \\ \hline
\textbf{SI-12}  & information handling and retention                          & Basso             & A             \\ \hline
\textbf{SI-16}  & memory protection                                           & Basso             & A             \\ \hline
\end{ltabulary}
\begin{center}
\captionof{table}{Controlli della classe SI} 
\end{center}

I controlli appartenenti a questa classe mirano a verificare la corretta funzionalità dei meccanismi per la preservazione dell'integrità dei sistemi e delle informazioni. Sono perlopiù automatizzabili, tuttavia potrebbe essere necessaria un'interazione umana per l'esecuzione dei controlli \textit{SI-2}, \textit{SI-2(2)} e \textit{SI-2(3)}, riguardanti le tempistiche di risoluzione di problematiche di sicurezza.


\subsection{Conclusioni della fase di analisi}
Dall'analisi effettuata si evince che dei 325 controlli delle \textit{baseline} bassa e moderata, 130 sono completamente automatizzabili (40\%), 133 richiedono interazione umana (41\%) e 62 richiedono un approccio misto tra automazione e interazione umana (19\%).
Molti di questi controlli, inoltre, sono relativi ad attività di monitoraggio continuativo, pertanto possono essere soddisfatti con la semplice adozione di \textit{Moon Cloud} o altre soluzioni di monitoraggio.
Si andrà ora a definire due metodologie per la copertura di tutti quei controlli di carattere procedurale, tecnico ed operativo.
La prima metodologia consiste nella gestione dei controlli automatizzabili, mediante l'integrazione di Moon Cloud con il protocollo SCAP; la seconda, invece, prevede la somministrazione di questionari agli attori coinvolti nei controlli di tipo procedurale ad interazione umana.

\section{Implementazione dei controlli automatici}
Per implementare i controlli automatizzabili di FedRAMP in Moon Cloud, è stato creato un driver con il supporto al protocollo \textit{SCAP}.
SCAP è una \textit{suite} composta da sei specifiche il cui obiettivo è standardizzare, utilizzando documenti XML, il formato e la nomenclatura con cui i prodotti di sicurezza comunicano informazioni relative alle configurazioni e alle vulnerabilità dei software noti, a umani e macchine.
SCAP utilizza alcuni documenti e dataset di riferimento (es. \textit{NVD, National Vulnerability Database}), forniti e mantenuti dal NIST e dal DHS.
Le finalità di SCAP sono molteplici:
\begin{itemize}
    \item Massimizzare l'interoperabilità tra strumenti di tipologia diversa
    \item Favorire la software integration tra soluzioni di produttori differenti
    \item Favorire lo sviluppo collaborativo dei documenti e dei dataset di riferimento
\end{itemize}
following five categories (with overview of the components):
languages – XCCDF, OVAL, OCIL,
• reporting formats – ARF, Asset Identification
• enumerations – CPE, CCE, CVE,
• measurement and scoring systems – CVSS, CCSS and
• integrity – TMSAD.
Many of these components use XML as the underlying technology
and a typical SCAP content then consists of multiple pieces of XML
with format defined by the (standalone) specifications of the components
listed above either in separate, but mutually linked, files or in
a single file with the pieces encapsulated in data stream collection
defined in the aforementioned NIST Special Publication 800-126.
Every well-formed [10] XML file must have one root element. In
case of the SCAP content encapsulated in a single file it is the datastream-collection
element that defines XML namespaces and
is composed of one or more data-streams, components and optionally
Signatures. Signature elements (which are required to
be valid XML digital signatures [11]) can be used to ensure integrity
and authenticity of the data. component elements then hold the content,
as defined by the SCAP components’ specifications, that are referenced
by the data-streams. data-stream elements are composed
(among the other things) of dictionaries (links to CPE
dictionaries), checklists (links to XCCDF content) and checks
(links to OVAL or OCIL checks). Since the intention was to facilitate
the creation of data streams and data stream collections from
an existing content using separate files, all these links are realized
as component-ref elements that provide IDs for what used to be
separate files (so that all the references from an existing content are
still valid) and contain catalogs that do basically the same for the
referenced component – i.e. catalogs define mappings between
what used to be separate files used by the component to IDs of the
component-refs for the content from those files.
The rest of the chapter is dedicated to descriptions of the SCAP
components, that are the basic building blocks of the SCAP content,
with the amounts of text proportional to the importance of the components
in connection with this thesis.

%languages
The languages category contains the two probably most important
components of SCAP – Extensible Configuration Checklist Description
Format (XCCDF) and Open Vulnerability and Assessment Language
(OVAL) – and the Open Checklist Interactive Language (OCIL).

%xccdf
In short, the Extensible Configuration Checklist Description Format
is a language that can be used to define rules and cases in which
particular rules should be applied
The XCCDF provides a way how to define rules, but only on
the level of their titles, descriptions and IDs. While for the human
reader just a description (or even a title) like “Create a separate partition
or logical volume for /tmp” could be enough to get the meaning
of the rule and to perform a check, for a computer anything like
that is practically impossible. And since one of the main goals of the
SCAP is automation, it needs to provide a way how to include some
machine-readable data to the rules on basis of which an automatic
evaluation could be done.

%oval
That is where the Open Vulnerability and Assessment Language (OVAL) comes into play with its three major
components – OVAL System Characteristics, OVAL Definitions and
OVAL Results as defined by The OVAL Language Specification [
Examples of such extending specifications are The OVAL
Language UNIX Component Model Specification [13] for UNIX-like
systems [14] and The OVAL Language Windows Component Model
Specification [15] for Windows systems, but these are definitely not
the only ones (in total there are currently 15 extensions for OVAL
Definitions). An interesting fact about The OVAL Language Speci-
fication and the specifications of the extensions is that they use the
Unified Modeling Language (UML) [16] to describe XML elements.
On one hand it is quite unusual, but on the other hand it perfectly
fits the model the OVAL language uses where new elements are in
many cases defined as being inherited2
from some other elements
and elements often use (contain) the other elements.
From the perspective of this thesis the most important part of
the OVAL language are Oval Definitions that are, among the other
things, used in XCCDF rules’ checks. Example of OVAL definitions
can be seen in the attachments. It contains metadata about the content
generation (product\_name, schema\_version and timestamp
elements) followed by the definitions, tests and objects. Each
definition has some metadata (title, description) and criteria
where each criterion references one or more tests that describe
what should hold for some objects. The basic framework covered
by the OVAL language specification consists of all the elements but
tests and objects that are system-specific (and thus specified in
the extensions that apply to a particular system).

As it was already mentioned above, an OVAL content is, among
the other things, usually used to define checks whether XCCDF rules
are followed or not. That means there has to be an interpreter that
actually checks the system according to the OVAL content and provides
back data showing if the check failed of passed. There is a freely
available3
referential implementation of such tool – The OVAL Interpreter
[18], but since it doesn’t cover all the functionality needed
in many cases (especially the extensions), some other interpreters
are usually being used. For example there is the jOVAL [19] project
focused on implementing a more comprehensive OVAL interpreter
(covering more extensions) that is licensed under the Affero GPL license
[20].
Another important part of the OVAL framework is the OVAL
Repository described on its home page as follows:
“The OVAL Repository is the central meeting place for
the OVAL Community to discuss, analyze, store, and
disseminate OVAL Definitions. Members of the community
contribute definitions by posting them to the
OVAL Repository Forum, where the OVAL Team and
other members of the community review and discuss
them.” [21]
which plays an important role in reusability of the OVAL content.
Though it may seem that the OVAL covers everything needed
to define checks for XCCDF rules, there are cases when it cannot
be used because of the fact that it is oriented only on systems and
their states and configurations and not on the users of such systems.
This allows OVAL content to be evaluated automatically, but
on the other hand doesn’t cover the rules like “All users must have
passed the basic security training.” that may appear in the SCAP content.
For cases like that elements of the Open Checklist Interactive
Language are used. The most recent format of the language is de-
fined by the NIST Interagency Report 7692 – Specification for the
Open Checklist Interactive Language (OCIL) Version 2.0. [22] The
OCIL provides a unified and general framework for creating questionnaires
that can be used, among the other places4
, in a SCAP con-


\subsection{Il framework OpenSCAP}
An important feature of the SCAP, also pointed out in it’s name, is
that it allows automation which means that the evaluation of a SCAP
content on a particular system can be done by some tool instead of
manual verification of the rules and manual fixing of the system’s
state and configuration. On the other hand the creation of SCAP content
cannot be done automatically (as today’s computers are not able
to detect a vulnerability, provide a fix for it and compose an XML file
will all the data needed), but still can be done in a “computer-aided”
way. There are many projects focused on both SCAP content creation
and evaluation, some of them focusing on separate particular components
and some of them trying to cover the whole specification.
Fortunately, many projects are focused on producing open-source [4]
tools and content that is publicly available with all the source code
for usage, modifications and redistribution which means that they
could be used as part of the Fedora GNU/Linux distribution strictly
requiring all its components to be open-source and publicly available.
We have already seen examples of such projects in the section
describing the OVAL (2.1) – OVAL Interpreter and jOVAL – both focused
on the evaluation of the OVAL content.
An example of a project that tries to cover all components of the
SCAP is the OpenSCAP project. Although still being under development
it already provides powerful tools to process the SCAP content
and creates an important “crossroad” for many other projects focused
on particular components. The OpenSCAP per se provides [3]:
• a library that can be used for SCAP content processing and
evaluation,
a scanner that utilizes the library and provides local scanning
capabilities,
• a number of XSLT1
[38] transformations that can be used to
transform an XML content to more human-readable HTML
format and
• SCAP content intended for testing and experimental purposes.
The library is written in C, but there are bindings for Perl and Python
languages, so that it can be used also from these high-level languages.
Probably the most important part of the OpenSCAP project is the
scanner (called oscap) which can be used for a wide range of actions
done with a SCAP content starting from validation and basic
information extraction going through various transformations and
ending with complete evaluation. A

\subsection{Driver OpenSCAP per Moon Cloud}
\subsection{OpenSCAP per la FedRAMP readiness}
\subsection{OpenSCAP per la NIST 800-53}
\section{Controlli ad interazione umana}
\subsection{Questionari per l'assessment dei controlli procedurali}
\subsection{Templating del questionario}
\end{document}
