\documentclass[../main.tex]{subfiles}
\begin{document}
\chapter{Implementazione dei controlli di sicurezza FedRAMP in Moon Cloud}
%\addcontentsline{toc}{chapter}{Introduzione}
%\chaptermark{Introduzione}i
\section{Introduzione}
In questo capitolo verrà effettuata un'analisi dei controlli di sicurezza elencati nel capitolo precedente, classificandoli in \textit{controlli automatici} e \textit{controlli procedurali}. Dopodiché verra offerta una possibile implementazione per ciascuna delle due tipologie di controlli, i quali verranno integrati in Moon Cloud. Per i controlli automatici sarà utilizzato Open Scap, uno strumento per il l'auditing realizzato da Red Hat e certificato dal NIST. I controlli procedurali invece, eseguono l'\textit{processi di business} e vanno ad indirizzare tutte quelle proprietà di carattere puramente qualitativo per cui è fondamentale l'interazione umana.
Questi saranno implementati con un \textit{driver Moon Cloud ad interazione umana}, che somministra un questionario online ad un target.

\section{Analisi dei controlli di sicurezza}
\subsection{Access Control}
\section{Implementazione dei controlli automatici}
\subsection{Il framework OpenSCAP}
\subsection{Driver OpenSCAP per Moon Cloud}
\subsection{OpenSCAP per la FedRAMP readiness}
\subsection{OpenSCAP per la NIST 800-53}
\section{Controlli ad interazione umana}
\subsection{Questionari per l'assessment dei controlli procedurali}
\subsection{Templating del questionario}
\end{document}
