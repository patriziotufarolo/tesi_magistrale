\documentclass[../main.tex]{subfiles}
\begin{document}
\chapter{Implementazione dei controlli di sicurezza FedRAMP in Moon Cloud}
%\addcontentsline{toc}{chapter}{Introduzione}
%\chaptermark{Introduzione}
\section{Introduzione}
In questo capitolo verrà effettuata un'analisi dei controlli di sicurezza elencati nel capitolo precedente, classificandoli in \textit{controlli automatici} e \textit{controlli procedurali}. Dopodiché verra offerta una possibile implementazione per ciascuna delle due tipologie di controlli, i quali verranno integrati in Moon Cloud. Per i controlli automatici sarà utilizzato Open Scap, uno strumento per il l'auditing realizzato da Red Hat e certificato dal NIST. I controlli procedurali invece, eseguono l'\textit{processi di business} e vanno ad indirizzare tutte quelle proprietà di carattere puramente qualitativo per cui è fondamentale l'interazione umana.
Questi saranno implementati con un \textit{driver Moon Cloud ad interazione umana}, che somministra un questionario online ad un target.

\section{Analisi dei controlli di sicurezza}
Verranno ora proposte alcune tabelle di riepilogo dei controlli di sicurezza di FedRAMP, suddivisi per famiglia, dei quali viene indicata la \textit{baseline} di riferimento e la tipologia (\textit{A}, automatizzabile / \textit{M} manuale (interazione umana) / \textit{A/M} composizione di approccio automatico e manuale}.
I controlli di sicurezza, essendo proposti in modo astratto così da poter coprire il più vasto numero di scenari, vanno poi comunque studiati e implementati sulla base delle caratteristiche e del contesto del sistema target.
\subsection{Access Control}
%\makeatletter
\begin{ltabulary}{|p{2cm}|p{8cm}|L|p{1cm}|}
  \toprule
    \hline
    \textbf{ID}    & \textbf{Nome}                                                        & \textbf{Baseline} & \textbf{Tipo} \\ \hline
  \midrule
  \endhead
  \textbf{AC-1 }   & access control policy and procedures                                 & Basso             & P             \\ \hline
  \textbf{AC-2 }   & account management                                                   & Basso             & A/P           \\ \hline
AC-2 (1)           & automated system account management                                  & Moderato          & A/P           \\ \hline
AC-2 (2)           & removal of temporary / emergency accounts                            & Moderato          & A             \\ \hline
AC-2 (3)           & disable inactive accounts                                            & Moderato          & A/P           \\ \hline
AC-2 (4)           & automated audit actions                                              & Moderato          & A             \\ \hline
AC-2 (5)           & inactivity logout                                                    & Moderato          & A             \\ \hline
AC-2 (7)           & role-based schemes                                                   & Moderato          & P             \\ \hline
AC-2 (9)           & restrictions on use of shared groups / accounts                      & Moderato          & A             \\ \hline
AC-2 (10)          & shared / group account credential termination                        & Moderato          & A/P           \\ \hline
AC-2 (12)          & account monitoring / atypical usage                                  & Moderato          & P             \\ \hline
\textbf{AC-3 }     & access enforcement                                                   & Basso             & A/P           \\ \hline
\textbf{AC-4 }     & information flow enforcement                                         & Moderato          & A/P           \\ \hline
AC-4 (21)          & physical / logical separation of information flows                   & Moderato          & A/P           \\ \hline
\textbf{AC-5 }     & separation of duties                                                 & Moderato          & A/P           \\ \hline
\textbf{AC-6 }     & least privilege                                                      & Moderato          & A/P           \\ \hline
AC-6 (1)           & authorize access to security functions                               & Moderato          & A             \\ \hline
AC-6 (2)           & non-privileged access for nonsecurity functions                      & Moderato          & A             \\ \hline
AC-6 (5)           & privileged accounts                                                  & Moderato          & A             \\ \hline
AC-6 (9)           & auditing use of privileged functions                                 & Moderato          & A             \\ \hline
AC-6 (10)          & prohibit non-privileged users from execution of privileged functions & Moderato          & A             \\ \hline
\textbf{AC-7 }     & unsuccessful logon attempts                                          & Basso             & A             \\ \hline
\textbf{AC-8 }     & system use notification                                              & Basso             & A             \\ \hline
\textbf{AC-10}     & concurrent session control                                           & Moderato          & A             \\ \hline
\textbf{AC-11}     & session lock                                                         & Moderato          & A             \\ \hline
AC-11 (1)          & pattern-hiding displays                                              & Moderato          & A             \\ \hline
\textbf{AC-12}     & session termination                                                  & Moderato          & A             \\ \hline
\textbf{AC-14}     & "permitted actions without identification or authentication"         & Basso             & A/P           \\ \hline
\textbf{AC-17}     & remote access                                                        & Basso             & A/P           \\ \hline
AC-17 (1)          & automated monitoring / control                                       & Moderato          & P             \\ \hline
AC-17 (2)          & protection of confidentiality / integrity using encryption           & Moderato          & A/P           \\ \hline
AC-17 (3)          & managed access control points                                        & Moderato          & A             \\ \hline
AC-17 (4)          & privileged commands / access                                         & Moderato          & A/P \ \hline
AC-17 (9)          & disconnect / disable access                                          & Moderato          & A/P           \\ \hline
\textbf{AC-18}     & wireless access                                                      & Basso             & A/P           \\ \hline
AC-18 (1)          & authentication and encryption                                        & Moderato          & A/P           \\ \hline
\textbf{AC-19}     & access control for mobile devices                                    & Basso             & A             \\ \hline
AC-19 (5)          & full device / container-based encryption                             & Moderato          & A             \\ \hline
\textbf{AC-20}     & use of external information systems                                  & Basso             & A             \\ \hline
AC-20 (1)          & limits on authorized use                                             & Moderato          & A             \\ \hline
AC-20 (2)          & portable storage devices                                             & Moderato          & A             \\ \hline
\textbf{AC-21}     & information sharing                                                  & Moderato          & P             \\ \hline
    \textbf{AC-22} & publicly accessible content                                          & Basso             & A/P           \\ \hline
\end{ltabulary}
%AC17 1
%\makeatother

Il principio seguito è il \textit{least-privilege}, che consiste nel concedere a ciascun utente i privilegi minimi di accesso necessari esclusivamente al compimento delle azioni di suo interesse.
Abbiamo un totale di 43 controlli, di cui 22 sono completamente automatizzabili, 5 sono di carattere procedurale e necessitano interazione umana, mentre per 16 di questi è necessario adottare un approccio ibrido.
Pertanto possiamo utilizzare un approccio totalmente automatico solo nel 51.1\%  dei casi, tuttavia considerando anche i casi in cui è richiesta sia la modalità automatica che quella manuale, la percentuale sale all'88\%.
Bisogna però notare che nel caso sia utilizzato Moon Cloud per monitorare i processi di autenticazione e autorizzazione, i controlli AC-6 (9) e AC-17 (1) possono essere automaticamente soddisfatti.

\section{Implementazione dei controlli automatici}
\subsection{Il framework OpenSCAP}
\subsection{Driver OpenSCAP per Moon Cloud}
\subsection{OpenSCAP per la FedRAMP readiness}
\subsection{OpenSCAP per la NIST 800-53}
\section{Controlli ad interazione umana}
\subsection{Questionari per l'assessment dei controlli procedurali}
\subsection{Templating del questionario}
\end{document}
