\documentclass[../main.tex]{subfiles}
\begin{document}
\chapter{Validazione del framework}
In questo capitolo verrà proposta una validazione di un deployment di test del framework Moon Cloud, mediante la valutazione della sicurezza tramite l'esecuzione del driver sviluppato a tutti i livelli dello stack cloud (infrastruttura, piattaforma, software). 

\section{Deployment di Moon Cloud}

Il deployment dell'ambiente di test è stato effettuato su un'architettura multi-layer così composta:
\begin{itemize}
    \item A livello di infrastruttura, un nodo fisico, Dell PowerEdge T430 equipaggiato con processore Intel(R) Xeon(R) CPU E5-2630 v4 (10 core fisici e 40 thread logici), 48GB di RAM e 5 dischi da 1TB ciascuno in configurazione RAID 1. Su questa macchina è stato installato il sistema operativo CentOS 7.2, e il software Open Stack Newton.
        Successivamente sono state create 3 macchine virtuali con sistema operativo CentOS 7.2, in un flavor con 4GB di RAM e 10GB di disco.
    \item A livello di piattaforma, un cluster Docker Swarm di 2 nodi per i componenti core di Moon Cloud. La terza macchina virtuale è stata utilizzata come nodo di esecuzione per un cluster di dimensione unaria.
    \item A livello software, i componenti di Moon Cloud: 7 servizi core, gestiti tramite container Docker sul cluster swarm e 3 microservizi per la gestione del nodo di esecuzione.
        \textbf{Componenti Core}
        \begin{itemize}
            \item API
            \item Database
            \item Repository
            \item Database \textit{time-series} per i risultati
            \item Evaluation Manager
            \item RabbitMQ per la comunicazione tra API e Container
            \item Traefik (reverse proxy) per l'esposizione delle API
        \end{itemize}
        \textbf{Componenti execution node e execution cluster}
        \begin{itemize}
            \item Execution Manager
            \item Monitor Execution Manager
            \item Traefik (reverse proxy) per l'esposizione di alcune tipologie di test
        \end{itemize}
\end{itemize}


\section{Sicurezza del deployment}
L'analisi della sicurezza è stata effettuata a livello di infrastruttura, analizzando le configurazioni sul sistema operativo del server fisico e le caratteristiche del setup OpenStack; a livello di piattaforma, analizzando le configurazioni del template utilizzato per le macchine virtuali Docker e i container Moon Cloud; a livello applicativo, analizzando la sicurezza nei meccanismi di integrazione dei vari componenti.
\subsection{Infrastruttura}
\subsubsection{Analisi OpenSCAP del server fisico}

test finished
started at 1496183480
ended at 1496184352
lasted -872 seconds

report available at
https://moonclouddashboard.blob.core.windows.net/pdfcontainer/f4a1943cff

\subsubsection{Analisi di OpenStack}

\subsection{Piattaforma}
\subsubsection{Analisi OpenSCAP sul template delle VM}
\subsubsection{Analisi dei container Moon Cloud}

\subsection{Applicazione}
\subsubsection{Analisi dei componenti software}

\end{document}
