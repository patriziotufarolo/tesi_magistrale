\documentclass[../main.tex]{subfiles}
\begin{document}
\chapter{Validazione del framework}
In questo capitolo verrà proposta una validazione di un deployment di test del framework Moon Cloud, mediante la valutazione della sicurezza tramite l'esecuzione del driver sviluppato a tutti i livelli dello stack cloud (infrastruttura, piattaforma, software). 

\section{Deployment di Moon Cloud}

Il deployment dell'ambiente di test è stato effettuato su un'architettura multi-layer così composta:
\begin{itemize}
    \item A livello di infrastruttura, un nodo fisico, Dell PowerEdge T430 equipaggiato con processore Intel(R) Xeon(R) CPU E5-2630 v4 (10 core fisici e 40 thread logici), 48GB di RAM e 5 dischi da 1TB ciascuno in configurazione RAID 1. Su questa macchina è stato installato il sistema operativo CentOS 7.2, e il software Open Stack Newton.
        Successivamente sono state create 3 macchine virtuali con sistema operativo CentOS 7.2, in un flavor con 4GB di RAM e 10GB di disco.
    \item A livello di piattaforma, un cluster Docker Swarm di 2 nodi per i componenti core di Moon Cloud. La terza macchina virtuale è stata utilizzata come nodo di esecuzione per un cluster di dimensione unaria.
    \item A livello software, i componenti di Moon Cloud: 7 servizi core, gestiti tramite container Docker sul cluster swarm e 3 microservizi per la gestione del nodo di esecuzione.\\
        \textbf{Componenti Core}
        \begin{itemize}
            \item API
            \item Database
            \item Repository
            \item Database dei risultati
            \item Evaluation Manager
            \item Meccanismo di comunicazione basato su code
            \item Reverse proxy
        \end{itemize}
        \textbf{Componenti execution node e execution cluster}
        \begin{itemize}
            \item Execution Manager
            \item Monitor Execution Manager
            \item Traefik (reverse proxy) per l'esposizione del servizio web \textit{Survey}
        \end{itemize}
\end{itemize}


\section{Sicurezza del deployment}
L'analisi della sicurezza è stata effettuata a livello di infrastruttura, analizzando le configurazioni sul sistema operativo del server fisico e le caratteristiche del setup OpenStack; a livello di piattaforma, analizzando le configurazioni del template utilizzato per le macchine virtuali Docker e i container Moon Cloud; a livello applicativo, analizzando la sicurezza nei meccanismi di integrazione dei vari componenti.
\subsection{Infrastruttura}
\subsubsection{Analisi OpenSCAP del server fisico}
\begin{comment}
test finished
started at 1496183480
ended at   1496184352
lasted -872 seconds
report available at
https://moonclouddashboard.blob.core.windows.net/pdfcontainer/f4a1943cff

\end{comment}
Il test è stato eseguito utilizzando il driver realizzato, specificando in input il documento XCCDF "\textit{ssg-centos7}" unitamente al profilo "\textit{nist-800-171-cui}".
Il documento \textit{JSON} dato in input al test è il seguente:
\begin{js}
{
    "read_ssh_configuration": {
    },
    "read_xccdf_configuration": {
        "xccdf":"ssg-centos7",
        "profile":"nist-800-171-cui",
        "fetch_remote_resources": true
    },
    "read_azure_configuration": {
    }
}
\end{js}

I risultati restituiti in output segnalano:
\begin{itemize}
    \item 119 valutazioni effettuate con successo
    \item 72 valutazioni fallite
\end{itemize}
\begin{figure}[H]
    \centering
    \includegraphics[width=15cm]{immagini/test_oscap_1.png}
\end{figure}

In particolare, delle 72 valutazioni fallite, 11 sono state catalogate da OpenSCAP con \textit{severity} bassa, 50 con \textit{severity} media e 11 con \textit{severity} elevata; lo score della copertura XCCDF ottenuto è 71.17\%.
\vfill\newpage
\subsubsection{Risultato del test: valutazioni fallite}
\begin{ltabulary}{|p{10cm}|p{4cm}|}
    \hline
    \textbf{Descrizione} & \textbf{Security Control}
    \\ \hline
    \endhead

Allow Only SSH Protocol 2                                                             & AC-17, IA-5 \\ \hline
Build and Test AIDE Database                                                          & CM-3(d), CM-3(e), CM-6(d), CM-6(3), SC-28, SI-7 \\ \hline
Configure AIDE to Use FIPS 140-2 for Validating Hashes                                & SI-7(1) \\ \hline
Configure AIDE to Verify Access Control Lists (ACLs)                                  & SI-7.1 \\ \hline
Configure AIDE to Verify Extended Attributes                                          & SI-7.1 \\ \hline
Configure Notification of Post-AIDE Scan Details                                      & CM-3(5)  \\ \hline
Configure Periodic Execution of AIDE                                                  & CM-3(d), CM-3(e), CM-3(5), CM-6(d), CM-6(3), SC-28, SI-7 \\ \hline
Configure the root Account for failed Password Attempts                               & AC-7(b) \\ \hline
Direct root Logins Not Allowed                                                        & IA-2(1) \\ \hline
Disable Compression Or Set Compression to delayed                                     & CM-6(b) \\ \hline
Disable Ctrl-Alt-Del Reboot Activation                                                & AC-6 \\ \hline
Disable GSSAPI Authentication                                                         & CM-6(c) \\ \hline
Disable KDump Kernel Crash Analyzer (kdump)                                           & AC-17(8), CM-7, CM-6(b) \\ \hline
Disable Kerberos Authentication                                                       & CM-6(c) \\ \hline
Disable Prelinking                                                                    & CM-6(c) \\ \hline
Disable SSH Access via Empty Passwords                                                & AC-3, AC-6, CM-6(b) \\ \hline
Disable SSH Root Login                                                                & AC-3, AC-17(2), AU-10(5), CM-6(b), IA-5(1)(c), IA-7 \\ \hline
Disable SSH Support for Rhosts RSA Authentication                                     & AC-3, AC-6, CM-6(b) \\ \hline
Disable SSH Support for User Known Hosts                                              & CM-6(a) \\ \hline
Disable Support for RPC IPv6                                                          & CM-6(a) \\ \hline
Disable xinetd Service                                                                & AC-17(8), CM-7 \\ \hline
Do Not Allow SSH Environment Options                                                  & CM-6(b)  \\ \hline
Enable Smart Card Login                                                               & IA-2(2) \\ \hline
Enable SSH Warning Banner                                                             & AC-8(a), AC-8(b), AC-8(c)(1), AC-8(c)(2), AC-8(c)(3) \\ \hline
Enable Use of Privilege Separation                                                    & AC-6 \\ \hline
Enable Use of Strict Mode Checking                                                    & AC-6 \\ \hline
Ensure gpgcheck Enabled For All Yum Package Repositories                              & CM-5(3), SI-7, MA-1(b) \\ \hline
Ensure gpgcheck Enabled for Local Packages                                            & CM-5(3) \\ \hline
Ensure gpgcheck Enabled for Repository Metadata                                       & CM-5(3) \\ \hline
Ensure Logs Sent To Remote Host                                                       & AU-3(2), AU-4(1), AU-9 \\ \hline
Ensure System is Not Acting as a Network Sniffer                                      & CM-7, CM-7(2).1(i) \\ \hline
Ensure the Logon ure Delay is Set Correctly in login.defs                             & IA-5(f), IA-5(1)(a) \\ \hline
Ensure YUM Removes Previous Package Versions                                          & SI-2 \\ \hline
Install AIDE                                                                          & CM-3(d), CM-3(e), CM-6(d), CM-6(3), SC-28, SI-7 \\ \hline
Install the dracut-fips Package                                                       & AC-17(2) \\ \hline
Install Virus Scanning Software                                                       & SC-28, SI-3 \\ \hline
Limit Password Reuse                                                                  & IA-5(1)(e) \\ \hline
Limit the Number of Concurrent Login Sessions Allowed Per User                        & AC-10  \\ \hline
Modify the System Login Banner                                                        & AC-8(a), AC-8(b), AC-8(c)(1), AC-8(c)(2), AC-8(c)(3) \\ \hline %
Prevent Log In to Accounts With Empty Password                                        & AC-6, IA-5(b), IA-5(c), IA-5(1)(a) \\ \hline
Restrict Serial Port Root Logins                                                      & AC-6(2) \\ \hline
Restrict Virtual Console Root Logins                                                  & AC-6(2) \\ \hline
Set Account Expiration Foling Inactivity                                              & AC-2(2), AC-2(3), IA-4(e) \\ \hline
Set Deny For failed Password Attempts                                                 & AC-7(b) \\ \hline
Set Interactive Session Timeout                                                       & AC-12, SC-10 \\ \hline
Set Interval For Counting ed Password Attempts                                        & AC-7(b) \\ \hline
Set Lockout Time For ed Password Attempts                                             & AC-7(b) \\ \hline
Set Password Maximum Age                                                              & IA-5(b), IA-5(c), IA-5(1)(a) \\ \hline
Set Password Maximum Consecutive Repeating Characters                                 & IA-5(b), IA-5(c), IA-5(1)(a) \\ \hline
Set Password Minimum Age                                                              & IA-5(b), IA-5(c), IA-5(1)(a) \\ \hline
Set Password Minimum Length                                                           & IA-5(b), IA-5(c), IA-5(1)(a) \\ \hline
Set Password Minimum Length in login.defs                                             & IA-5(b), IA-5(c), IA-5(1)(a) \\ \hline
Set Password Retry Prompts Permitted PSession                                         & IA-5(b), IA-5(c), IA-5(1)(a) \\ \hline
Set Password Strength Minimum Different Categories                                    & IA-5(b), IA-5(c), IA-5(1)(a) \\ \hline
Set Password Strength Minimum Different Characters                                    & IA-5(b), IA-5(c), IA-5(1)(a) \\ \hline
Set Password Strength Minimum Digit Characters                                        & IA-5(b), IA-5(c), IA-5(1)(a) \\ \hline
Set Password Strength Minimum Special Characters                                      & IA-5(b), IA-5(c), IA-5(1)(a) \\ \hline
Set Password Strength Minimum Uppercase Characters                                    & IA-5(b), IA-5(c), IA-5(1)(a) \\ \hline
Set Password to Maximum of Consecutive Repeating Characters from Same Character Class & IA-5(b), IA-5(c), IA-5(1)(a) \\ \hline
Set SSH Client Alive Count                                                            & AC-2(5), SA-8, AC-12 \\ \hline
Set SSH Idle Timeout Interval                                                         & AC-7(b) \\ \hline
The Installed Operating System Is Vendor Supported and Certified                      & SI-2(c) \\ \hline
Uninstall xinetd Package                                                              & AC-17(8), CM-7 \\ \hline
Use Only FIPS 140-2 Validated Ciphers                                                 & AC-3, AC-17(2), AU-10(5), CM-6(b), IA-5(1)(c), IA-7 \\ \hline
Use Only FIPS 140-2 Validated MACs                                                    & AC-17(2), IA-7, SC-13 \\ \hline
Verify and Correct File Permissions with RPM                                          & AC-6, AU-9(1), AU-9(3), CM-6(d), CM-6(3) \\ \hline
Verify File Hashes with RPM                                                           & CM-6(d), CM-6(3), SI-7(1) \\ \hline
Verify firewalld Enabled                                                              & CM-6(b) \\ \hline

\end{ltabulary}
\captionof{table}{Risultati valutazione driver OpenSCAP a livello IaaS} 
\begin{figure}[H]
    \centering
    \includegraphics[width=10cm]{immagini/test_oscap_1_1.png}
    \caption{Estratto del report}\label{ref:report_oscap_1_1}
\end{figure}
    
In figura \ref{ref:report_oscap_1_1} è riportata una parte del report generato dal controllo sviluppato. Esso, in formato HTML, contiene il riepilogo di tutti i test eseguiti con il relativo stato e gli script di \textit{remediation}. Il documento completo è disponibile in formato HTML all'indirizzo:\\
\textit{https://moonclouddashboard.blob.core.windows.net/pdfcontainer/f4a1943cff}
\\
o nel repository GIT\\
\textit{https://github.com/patriziotufarolo/tesi\_magistrale/}.

La valutazione delle \textit{OVAL} mediante il driver realizzato è durata 872 secondi; di seguito sono riportate le statistiche relative all'utilizzo delle risorse nell'esecuzione del test.
Queste sono state raccolte tramite il software \texttt{pidstat}, isolando esclusivamente i processi coinvolti nel processo di assessment.
\begin{figure}[H]
 \begin{minipage}[b]{6cm}
   \centering
   \includegraphics[width=6.6cm]{immagini/plot1cpu.png}
 \end{minipage}
 \hspace{2mm} \hspace{3mm}
 \begin{minipage}[b]{9cm}
  \centering
   \includegraphics[width=6.6cm]{immagini/plot1io.png}
 \end{minipage}
 \caption{Utilizzo della CPU e carico IO}\label{ref:plot1cpuio}
\end{figure}
Dalla figura \ref{ref:plot1cpuio} è possibile notare come l'impatto del test sulle prestazioni del target sia minimo, registrando picchi di carico sulla CPU inferiori all'0.06\% per tutta la durata dell'esecuzione. Anche dal punto di vista delle operazioni di input/output il test si è dimostrato assolutamente non invasivo, registrando picchi in lettura di pochi KB. 
\begin{figure}[H]
 \begin{minipage}[b]{6cm}
   \centering
   \includegraphics[width=6.6cm]{immagini/plot1mem.png}
 \end{minipage}
 \hspace{2mm} \hspace{3mm}
 \begin{minipage}[b]{9cm}
  \centering
   \includegraphics[width=6.6cm]{immagini/plot1mem2.png}
 \end{minipage}
 \caption{Utilizzo della memoria}\label{ref:plot1mem}
\end{figure}
In figura \ref{ref:plot1mem} è stato invece illustrato l'utilizzo della memoria durante l'esecuzione del test, che rimane sempre inferiore all'1\%. Anche in questo caso l'impatto prestazionale è minimo; il \textit{footprint} di \textit{OpenSCAP} e delle relative \textit{probes} non raggiunge i 5MB in memoria virtuale (\textit{VSZ}), rimanendo inoltre sotto i 500 KB in memoria RAM(\textit{RSS}).
Questo driver si presta particolarmente a scenari di esecuzione programmata e automatica, per effettuare monitoraggio continuativo, dimostrandosi adeguato per l'utilizzo in Moon Cloud.

\subsubsection{Analisi di OpenStack}
I documenti XCCDF forniti da OpenSCAP non sono risultati efficienti per l'assessment dei controlli di sicurezza per Open Stack, in quanto le definizioni \textit{OVAL} non sono state redatte
Questa sezione, pertanto, fa riferimento ad alcune raccomandazioni del paper "\textit{A Security Benchmark for OpenStack}" \cite{MyPaper}. Ogni raccomandazione recensita è mappata sui requisiti FedRAMP corrispondenti. A causa del fatto che l'implementazione dei controlli automatici per alcuni di queste non sono disponibili, l'analisi è stata fatta in modo manuale ispezionando le configurazioni.
Laddove è applicata la dicitura "Non applicabile", si intende che il controllo non è effettuabile nello specifico scenario valutato.
\begin{ltabulary}{|p{6cm}|p{4cm}|p{2cm}|}
    \hline
    \textbf{Reccomendation} & \textbf{FedRAMP SCs} & \textbf{Risultato} \\ \hline
    \endhead 
    {[R1]} Patch Levels                                                                              & SA-10(1), SI(3)                  & Passato         \\ \hline
    {[R2]} Create and Enforce Account and Password Management Policies                                & AC-2, AC-2(3), AC-6, AC-7, AC-9  & Fallito         \\ \hline
    {[R3]} Use a Central Directory for Authentication and Authorization.                              & AC-3, AC-17                      & Fallito         \\ \hline
    {[R4]} Configure Firewalls to Restrict Access                                                     & CM-6, CM-7                       & Passato         \\ \hline
    {[R5]} Use TLS/SSL where Possible                                                                 & AC-3, AC-17(2), AU-10(5), CM-6(b), IA-5(1)(c), IA-7 & Fallito  \\ \hline
    {[R6]} Do Not Use Default Self-Signed Certificates.                                               & AC-3, AC-17(2), AU-10(5), CM-6(b), IA-5(1)(c), IA-7 & Non Applicabile \\ \hline
    {[R7]} Configure Centralized Remote Logging                                                       & AU-3(2), AU-4(1), AU-9           & Fallito         \\ \hline
    {[R8]} Maintain Time Synchronization Services                                                     & AU-8                             & Passato         \\ \hline
    {[R11]} Use Templates to Deploy Virtual Machines                                                  & -                                & Passato         \\ \hline
    {[R13]} Disable MAC Address Changes and Promiscuous Mode on Guests                                & CM-7, CM-7(2).1(i)               & Passato         \\ \hline
    {[R14]} Ensure Network Isolation through VLANs                                                    & SC-2, SC-7, SC-8                 & Passato         \\ \hline
    {[R15]} Port Groups Should not be Configured to Reserved VLANs                                    & CM-7                                & Passato         \\ \hline
    {[R16]} Secure Access to Cloud Application Programming Interfaces                                 & CM-7                                & Fallito         \\ \hline
    {[R17]} Encrypt Data at Rest                                                                      & SC-28, SC-28 (1)                & Fallito         \\ \hline
    {[R18]} Establish Appropriate Resource Isolation                                                  & SC-39                           & Passato         \\ \hline
    {[R19]} Evaluate Denial of Service Scenarios and Mitigations                                      & SC-06                            & Fallito         \\ \hline
    {[R20]} Do Not Use or Set Guest Customization Passwords                                           & IA-5(7)                          & Passato         \\ \hline
    {[R22]} Audit Sensible and Configuration Files                                                    & AU-3, AU-8, AU-12                & Fallito         \\ \hline
    {[R23]} Storage Reliability                                                                       & CP-4, CP-9                       & Fallito         \\ \hline
    {[R24]} Data Remanence Avoidance                                                                  & SC-4                             & Fallito         \\ \hline
\end{ltabulary}
\captionof{table}{Valutazione della compliance FedRAMP nel setup OpenStack di Moon Cloud} 

Il 55\% delle valutazioni ha avuto esito negativo sull'ambiente di sviluppo, in quanto esso non è ovviamente predisposto per essere utilizzato in scenari di produzione.

\subsection{Piattaforma}
\subsubsection{Analisi OpenSCAP sul template delle macchine virtuali}
Il test con il driver OpenSCAP è stato ripetuto sulle macchine virtuali dei nodi appartenenti al cluster Swarm.

I risultati restituiti in output segnalano:
\begin{itemize}
    \item 122 valutazioni effettuate con successo
    \item 69 valutazioni fallite
\end{itemize}
\begin{figure}[H]
    \centering
    \includegraphics[width=15cm]{immagini/test_oscap_2.png}
\end{figure}

Il report completo è disponibile all'indirizzo \\
\textit{https://moonclouddashboard.blob.core.windows.net/pdfcontainer/b73210ae24}    \\   
oppure \\
\textit{https://www.github.com/patriziotufarolo/tesi\_magistrale/} \\


\begin{comment}
test finished
started at 1496218136
ended at 1496218350
lasted -214 seconds

report available at
https://moonclouddashboard.blob.core.windows.net/pdfcontainer/b73210ae24
--
\end{comment}
\subsubsection{Analisi del cluster Docker Swarm}
\label{ref:dockerswarmanalysis}
L'analisi del cluster Swarm è stata condotta utilizzando lo strumento \textit{Docker Security Benchmark}\footnote{Docker Security Benchmark - \textit{https://www.dockerbench.com}}.
Di seguito è proposto un mapping tra le verifiche fallite e i controlli di sicurezza FedRAMP non rispettati; per motivi di proprietà intellettuale i nomi di alcuni container sono stati oscurati.
    \begin{ltabulary}{|p{9cm}|p{5cm}|}
        \hline
        \textbf{Controllo} & \textbf{FedRAMP SCs} \\
        \hline
        \endhead

Audit docker daemon - /usr/bin/docker                                                                                                                                                                                                                                                       & AU-3, AU-8, AU-12                                   \\ \hline
Audit Docker files and directories - /var/lib/docker, /etc/docker, docker.service, /docker-containerd, docker-runc                                                                                                                                                                          & AU-3, AU-8, AU-12                                   \\ \hline
Restrict network traffic between containers                                                                                                                                                                                                                                                 & SC-2, SC-7, SC-8                                    \\ \hline
Docker daemon currently listening on TCP without TLS                                                                                                                                                                                                                                        & AC-3, AC-17(2), AU-10(5), CM-6(b), IA-5(1)(c), IA-7 \\ \hline
Enable user namespace support                                                                                                                                                                                                                                                               & AC-4(21), AC-5, AC-6 (01), AC-6(02), AC-6(05)       \\ \hline
Use authorization plugin                                                                                                                                                                                                                                                                    & AC-3 AC-4                                                    \\ \hline
Configure centralized and remote logging                                                                                                                                                                                                                                                    & AU-3(2), AU-4(1), AU-9                           \\ \hline
Disable operations on legacy registry (v1)                                                                                                                                                                                                                                                  & -                                                 \\ \hline
Bind swarm services to a specific host interface                                                                                                                                                                                                                                            & SC-2, SC-7, SC-8, SC-10                                                     \\ \hline
Unencrypted overlay network: ingress (swarm), mooncloud (swarm)                                                                                                                                                                                                                             & SC-8, SC-8(1), SC-10                                                     \\ \hline
Run swarm manager in auto-lock mode                                                                                                                                                                                                                                                         & SC-27                                               \\ \hline
Create a user for the container                                                                                                                                                                                                                                                             &  AC-4(21), AC-5, AC-6 (01), AC-6(02), AC-6(05)                                                     \\ \hline
Running as root:  mooncloud\_evaluationmodule, mooncloud\_api, mooncloud\_traefik, mooncloud\_monitor\_execution\_manager, mooncloud\_dashboard, mooncloud\_*********, mooncloud\_***********, mooncloud\_db, mooncloud\_******, mooncloud\_execution\_manager                              &  AC-4(21), AC-5, AC-6 (01), AC-6(02), AC-6(05)                                                   \\ \hline
Enable Content trust for Docker:  mooncloud\_evaluationmodule, mooncloud\_api, mooncloud\_traefik, mooncloud\_monitor\_execution\_manager, mooncloud\_dashboard, mooncloud\_*********, mooncloud\_***********, mooncloud\_db, mooncloud\_******, mooncloud\_execution\_manager              &  AU-3(1), CM-07, SC-07, SC-08                                                   \\ \hline
No AppArmorProfile Found:   mooncloud\_evaluationmodule, mooncloud\_api, mooncloud\_traefik, mooncloud\_monitor\_execution\_manager, mooncloud\_dashboard, mooncloud\_*********, mooncloud\_***********, mooncloud\_db, mooncloud\_******, mooncloud\_execution\_manager                    & AC-3, AC-4, AC-5, AC-6, AU-5                    \\ \hline
No selinux SecurityOptions Found:  mooncloud\_evaluationmodule, mooncloud\_api, mooncloud\_traefik, mooncloud\_monitor\_execution\_manager, mooncloud\_dashboard, mooncloud\_*********, mooncloud\_***********, mooncloud\_db, mooncloud\_******, mooncloud\_execution\_manager             & AC-3, AC-4, AC-5, AC-6, AU-5                      \\ \hline
Container running with root FS mounted R/W:  mooncloud\_evaluationmodule, mooncloud\_api, mooncloud\_traefik, mooncloud\_monitor\_execution\_manager, mooncloud\_dashboard, mooncloud\_*********, mooncloud\_***********, mooncloud\_db, mooncloud\_******, mooncloud\_execution\_manager   & SI-12                                           \\ \hline
%SA-2
\end{ltabulary}



\subsubsection{Livello applicativo: analisi dei componenti software}

L'attività di assessment a livello applicativo è stata svolta ispezionando le configurazioni dei servizi e la loro implementazione, in un ambiente di sviluppo.
Per ragioni di proprietà intellettuale sarà di seguito fornita una descrizione ad alto livello dei componenti e delle problematiche individuate.

\begin{ltabulary}{|p{3cm}|p{11cm}|}
\hline
\textbf{Componente} & \textbf{Analisi} \\
\hline
\endhead

API e Database & È stato effettuato un \textit{vulnerability assessment} blackbox sfruttando i software \textit{SQLMap}\footnote{SQLMap, \textit{http://www.sqlmap.org}} e \textit{OpenVAS}\footnote{OpenVAS, \textit{http://www.openvas.org}} e non sono state riscontrate vulnerabilità note.
Le API sono esposte con protocollo \textit{HTTPS}, il certificato ha chiave \textit{RSA} a 4096 bit, è firmato con \textit{SHA256} ed è stato emesso dalla Certification Authority \textit{Let's Encrypt Authority X3}. Sono supportati i protocolli \textit{TLSv1}, \textit{TLSv1.1}, \textit{TLSv1.2}, il supporto a \textit{SSLv2} e \textit{SSLv3} è disabilitato per prevenire attacchi di downgrade del protocollo. Sono supportati due cifrari deboli, \textit{TLS\_ECDHE\_RSA\_ WITH\_3DES\_EDE\_ CBC\_SHA} e \textit{TLS\_RSA\_WITH\_ 3DES\_EDE\_CBC\_SHA}. L'esecuzione del test Moon Cloud per la verifica della connessione HTTPS ha riportato risultati positivi, restituendo il livello A.

L'autenticazione è basata su \textit{bearer token}, che vengono specificati in ogni richiesta nell'header HTTP "\textit{Authorization}"; l'utilizzo sicuro di questo approccio è vincolato a due fattori:
\begin{itemize}
    \item L'utilizzo di un canale cifrato in modo sicuro (verificato)
    \item L'applicazione di politiche di scadenza dei token (verificato)
\end{itemize}.

Le autorizzazioni sono gestite in modalità \textit{RBAC}\footnote{Role-based access control}; non sono applicate politiche sulla lunghezza né sulla complessità delle password; non sono presenti mecccanismi di rotazione delle stesse.

È stato riscontrato l'utilizzo di un server SMTP per l'invio delle e-mail senza autenticazione e cifratura dei dati in transito.

La connessione con il database non avviene in modo cifrato, né sono cifrati i dati memorizzati sullo stesso. Le password utente sono protette da hash \textit{pbkdf2\_sha256}\footnote{Password-Based Key Derivation Function}.
\\ \hline


Repository & Il repository è raggiungibile solamente da rete interna. La connessione è cifrata tramite HTTPS, con certificato emesso da "Let's Encrypt Authority X3", dalle caratteristiche analoghe al certificato API. Anche in questo caso, il test Moon Cloud per la verifica della connessione SSL ha riportato risultati positivi.

\\ \hline
 
Database dei risultati & Il database dei risultati è esposto tramite interfaccia HTTP, senza cifratura. Non sono presenti meccanismi di cifratura né in transito né \textit{at-rest}. L'autenticazione al database è effettuata tramite \textit{HTTP basic authentication}.
                       
\\ \hline

Evaluation Manager & È stato effettuato un \textit{vulnerability assessment} blackbox sfruttando i software \textit{SQLMap} e \textit{OpenVAS} e non sono state identificate vulnerabilità note.
Le API, il cui accesso è effettuato in chiaro, sono utilizzate solo localmente per lanciare le operazioni di valutazione. Il componente comunica con il database dei risultati e con il database API in chiaro. La comunicazione avviene in ogni caso su reti \textit{overlay} separate e cifrate.
Le credenziali di autenticazione al database API hanno privilegi minimi di accesso sulle sola tabelle relative alle valutazioni dei test.  \\ \hline

Execution Manager & L'execution manager riceve i dati in input dai test tramite un meccanismo di comunicazione basato su code. Su questo non è stata configurata la cifratura dei dati per proteggere la confidenzialità, né di firma volti a preservarne l'integrità.
I driver dei test sono eseguiti comunicando con un cluster Docker tramite socket Unix in chiaro; il download degli stessi viene fatto dal repository con connessione cifrata tramite SSL e autorizzata con modalità \textit{MAC}\footnote{Mandatory Access Control}. 
 \\ \hline
\end{ltabulary}

\begin{comment}
\section{Conclusioni}

È stato effettuato l'\textit{assessment} del setup \textit{multi-layer} di un ambiente di sviluppo per un'architettura a microservizi.

Per l'analisi del \textbf{livello infrastrutturale}, è stato utilizzato il driver sviluppato in questo lavoro di tesi, il cui obiettivo è quello di integrare le funzionalità di \textit{OpenSCAP} in \textit{Moon Cloud}. I risultati ottenuti, ampiamente caratterizzati in un report, hanno evidenziato varie carenze nella configurazione del sistema operativo della macchina fisica, soprattutto dal punto di vista della gestione delle politiche di accesso, la mancanza di meccanismi di \textit{trust} nei \textit{repository} dei pacchetti e l'assenza di tecnologie di auditing.
Per l'analisi del setup \textit{Open Stack}, ci si è basati sul contenuto dell'articolo "A security benchmark for Open Stack" \cite{MyPaper}, del quale è stato effettuato un mapping sui requisiti FedRAMP. I risultati ottenuti evidenziano ancora una volta mancanza di accuratezza nella configurazione degli aspetti di sicurezza. Tuttavia, poiché i sistemi installati sono dedicati all'utilizzo in laboratorio e non orientate alla produzione ed all'erogazione di servizi verso terzi, il livello di sicurezza effettivamente implementato è di qualità accettabile.

Per l'analisi del \textbf{livello di piattaforma}, è stato utilizzato nuovamente il driver \textit{OpenSCAP}, dopodiché è stato effettuato il \textit{benchmarking} della configurazione di sicurezza del cluster \textit{Docker Swarm}. Per ciascuna delle regole non validate, è stato effettuato il mapping sui requisiti FedRAMP violati.
I container Docker vengono eseguiti con utente \textit{root} e senza l'utilizzo di \textit{user namespace}, non sono presenti soluzioni \textit{mandatory access control} come \textit{AppArmor} o \textit{selinux}. 
 
L'analisi del \textbf{livello applicativo}, infine, si è svolta ispezionando i componenti dell'architettura \textit{Moon Cloud} e i meccanismi di comunicazione tra gli stessi. Sono state rilevate numerose carenze, alcune a livello di configurazione quindi sostenibili in un ambiente di sviluppo (appartenenti alla classe \textit{CM} di FedRAMP), altre a livello implementativo.
I canali di comunicazione tra alcuni componenti (coda per l'esecuzione dei test, connessione al database, comunicazione con il database dei risultati) sono in chiaro; considerando inoltre che le reti \textit{overlay} di \textit{Docker} non sono configurate in modo cifrato (come illustrato in sezione \ref{ref:dockerswarmanalysis}), il rischio di esposizione di informazioni sensibili è alto.

Il database a supporto delle API necessita di livello di sicurezza implementato a causa della tipologia dei dati trattati dalla piattaforma, come ad esempio i dati di autenticazione e le proprietà dei \textit{target} che devono essere verificati.
A tal proposito potrebbe essere integrato un \textit{vault} sicuro, crittografato con meccanismi e algoritmi \textit{FIPS compliant}.

Affinché il progetto Moon Cloud possa ricevere l'autorizzazione provvisoria ad operare dalla JBO FedRAMP, è necessario implementare le soluzioni ai problemi illustrati.
Per quanto riguarda l'aspetto infrastrutturale e di piattaforma, una possibilità è quella di rivolgersi a fornitori di servizi già autorizzati (Azure), come ad esempio Azure.
Le carenze del livello applicativo, invece, devono essere gestite in modo opportuno dal team di sviluppo applicando le \textit{remediation} opportune rispetto ai security control \textit{NIST 800-53}.
\end{comment}
\end{document}
