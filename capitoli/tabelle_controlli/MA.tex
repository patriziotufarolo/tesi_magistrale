\makeatletter

\begin{ltabulary}{|p{2cm}|p{8cm}|L|p{1cm}|}
    \hline
    \textbf{ID} & \textbf{Nome}                            & \textbf{Baseline} & \textbf{Tipo} \\ \hline
  \endhead
\textbf{MA-1}   & system maintenance policy and procedures & Basso             & P             \\ \hline
\textbf{MA-2}   & controlled maintenance                   & Basso             & P             \\ \hline
\textbf{MA-3}   & maintenance tools                        & Moderato          & A/P           \\ \hline
MA-3 (1)        & inspect tools                            & Moderato          & A             \\ \hline
MA-3 (2)        & inspect media                            & Moderato          & A             \\ \hline
MA-3 (3)        & prevent unauthorized removal             & Moderato          & A             \\ \hline
\textbf{MA-4}   & nonlocal maintenance                     & Basso             & P             \\ \hline
MA-4 (2)        & document nonlocal maintenance            & Moderato          & P             \\ \hline
\textbf{MA-5}   & maintenance personnel                    & Basso             & P             \\ \hline
MA-5 (1)        & individuals without appropriate access   & Moderato          & P             \\ \hline
\textbf{MA-6}   & timely maintenance                       & Moderato          & A/P           \\ \hline
\end{ltabulary}
\captionof{table}{Controlli della classe MA} 

Anche in questo caso ci troviamo in presenza di controlli di carattere perlopiù procedurale, nonostante l'approccio automatico possa essere utilizzato per garantire la correttezza delle operazioni svolte (ad esempio in \textit{MA-3(1)}, \textit{MA-3(2)} e \textit{MA-3(3)})

\makeatother
