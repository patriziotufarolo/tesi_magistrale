\begin{ltabulary}{|p{2cm}|p{8cm}|L|p{1cm}|}
    \hline
    \textbf{ID}     &\textbf{Nome}                                                          & \textbf{Baseline} & \textbf{Tipo}  \\    \hline
  \endhead
\textbf{MP-1} & media protection policy and procedures & Basso    & P \\ \hline
\textbf{MP-2} & media access                           & Basso    & P \\ \hline
\textbf{MP-3} & media marking                          & Moderato & P \\ \hline
\textbf{MP-4} & media storage                          & Moderato & P \\ \hline
\textbf{MP-5} & media transport                        & Moderato & A/P \\ \hline
MP-5 (4)      & cryptographic protection               & Moderato & A/P \\ \hline
\textbf{MP-6} & media sanitization                     & Basso    & P \\ \hline
MP-6 (2)      & equipment testing                      & Moderato & P \\ \hline
\textbf{MP-7} & media use                              & Basso    & A/P \\ \hline
MP-7 (1)      & prohibit use without owner             & Moderato & A/P \\ \hline
\end{ltabulary}
\begin{center}
\captionof{table}{Controlli della classe MP} 
\end{center}

Questa categoria di controlli riguarda perlopiù processi di business nei quali è implicato l'utilizzo di dispositivi multimediali per il trasferimento delle informazioni. Si tratta perciò di controlli che devono essere verificati con interazione umana, anche se possono essere effettuate alcune verifiche in modo automatico, come nel caso della proprietà \textit{MP-5} che verifica l'utilizzo della crittografia nel trasporto dei dati effettuato con dispositivi multimediali (CD, DVD, Pendrive).

