\makeatletter

\begin{ltabulary}{|p{2cm}|p{8cm}|L|p{1cm}|}
    \hline
    \textbf{ID}     &\textbf{Nome}                                                          & \textbf{Baseline} & \textbf{Tipo}  \\    \hline
  \endhead

    \textbf{AT-1}     & security awareness and training policy and procedures                  & Basso & P \\ \hline
    \textbf{AT-2}     & security awareness training                                           & Basso & P \\ \hline
    AT-2 (2)          & insider threat                                                        & Moderato & P \\ \hline
    \textbf{AT-3}     & role-based security training                                          & Basso & P \\ \hline
    \textbf{AT-4}     & security training records                                             & Basso & P \\ \hline
\end{ltabulary}
\captionof{table}{Controlli della classe AT} 
\makeatother
I controlli della famiglia \textit{awareness and training} sono soltanto 5, e per ciascuno di essi è richiesta intrinsecamente l'interazione umana.
Questa, infatti, misura il grado di consapevolezza dell'utente rispetto alle politiche e alle procedure di sicurezza. 

L'approccio ibrido può essere utilizzato qualora si voglia adottare un sistema automatico per la valutazione delle conoscenze dell'utente oppure l'individuazione di minacce provenienti dall'esterno.
