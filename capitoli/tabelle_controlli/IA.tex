\makeatletter

\begin{ltabulary}{|p{2cm}|p{8cm}|L|p{1cm}|}
    \hline
    \textbf{ID}     &\textbf{Nome}                                                          & \textbf{Baseline} & \textbf{Tipo}  \\    \hline
  \endhead

\textbf{IA-1} & "identification and authentication policy and procedures"                         & Basso    & P   \\ \hline
\textbf{IA-2} & "identification and authentication (organizational users)"                        & Basso    & P   \\ \hline
IA-2 (1)      & network access to privileged accounts                                             & Basso    & A   \\ \hline
IA-2 (2)      & network access to non-privileged accounts                                         & Moderato & A   \\ \hline
IA-2 (3)      & local access to privileged accounts                                               & Moderato & A   \\ \hline
IA-2 (5)      & "identification and authentication (organizational users) | group authentication" & Moderato & A   \\ \hline
IA-2 (8)      & network access to privileged accounts - replay resistant                          & Moderato & A   \\ \hline
IA-2 (11)     & remote access - separate device                                                   & Moderato & A   \\ \hline
IA-2 (12)     & acceptance of piv credentials                                                     & Basso    & A   \\ \hline
\textbf{IA-3} & device identification and authentication                                          & Moderato & A   \\ \hline
\textbf{IA-4} & identifier management                                                             & Basso    & P   \\ \hline
IA-4 (4)      & identify user status                                                              & Moderato & P   \\ \hline
\textbf{IA-5} & authenticator management                                                          & Basso    & A/P \\ \hline
IA-5 (1)      & password-based authentication                                                     & Basso    & A   \\ \hline
IA-5 (2)      & pki-based authentication                                                          & Moderato & A   \\ \hline
IA-5 (3)      & in-person or trusted third-party registration                                     & Moderato & A   \\ \hline
IA-5 (4)      & automated support for password strength determination                             & Moderato & A   \\ \hline
IA-5 (6)      & protection of authenticators                                                      & Moderato & A   \\ \hline
IA-5 (7)      & no embedded unencrypted static authenticators                                     & Moderato & A   \\ \hline
IA-5 (11)     & hardware token-based authentication                                               & Basso    & A   \\ \hline
\textbf{IA-6} & authenticator feedback                                                            & Basso    & A   \\ \hline
\textbf{IA-7} & cryptographic module authentication                                               & Basso    & A   \\ \hline
\textbf{IA-8} & identification and authentication (non- organizational users)                     & Basso    & A   \\ \hline
IA-8 (1)      & acceptance of piv credentials from other agencies                                 & Basso    & A   \\ \hline
IA-8 (2)      & acceptance of third-party credentials                                             & Basso    & A   \\ \hline
IA-8 (3)      & use of ficam-approved products                                                    & Basso    & A   \\ \hline
IA-8 (4)      & use of ficam-issued profiles                                                      & Basso    & A   \\ \hline
\end{ltabulary}
\captionof{table}{Controlli della classe IA} 

I controlli della famiglia di \textit{identificazione e autenticazione} sono perlopiù di carattere tecnico e pertanto in gran parte automatizzabili (nell'85\% dei casi).
Questi hanno una stretta correlazione con i controlli della famiglia \textit{controllo degli accessi} e di \textit{auditing}, già trattate, e manifestano la necessità per il provider di mantenere \textit{accountability} e \textit{non-repudiation}. Pertanto si tratta di controlli effettuati sulla resistenza delle credenziali agli attacchi, sulla sicurezza del processo di autenticazione e sulla possibilità di ricondurre con esattezza tutte le operazioni svolte sul sistema all'attore che le ha compiute, sia esso un servizio automatico o un utente umano.
Molti di questi, tuttavia, sono anche di carattere ibrido o procedurale, poiché basati sulle dichiarazioni del provider stesso.

\makeatother
