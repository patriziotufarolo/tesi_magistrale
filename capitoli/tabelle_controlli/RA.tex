\begin{ltabulary}{|p{2cm}|p{8cm}|L|p{1cm}|}
    \hline
    \textbf{ID} & \textbf{Nome}                                             & \textbf{Baseline} & \textbf{Tipo} \\ \hline
  \endhead
\textbf{RA-1}   & risk assessment policy and procedures                     & Basso             & P             \\ \hline
\textbf{RA-2}   & security categorization                                   & Basso             & P             \\ \hline
\textbf{RA-3}   & risk assessment                                           & Basso             & P             \\ \hline
\textbf{RA-5}   & vulnerability scanning                                    & Basso             & A/P           \\ \hline
RA-5 (1)        & update tool capability                                    & Moderato          & A             \\ \hline
RA-5 (2)        & update by frequency / prior to new scan / when identified & Moderato          & A             \\ \hline
RA-5 (3)        & breadth / depth of coverage                               & Moderato          & A/P           \\ \hline
RA-5 (5)        & privileged access                                         & Moderato          & A             \\ \hline
RA-5 (6)        & automated trend analyses                                  & Moderato          & A             \\ \hline
RA-5 (8)        & review historic audit logs                                & Moderato          & P             \\ \hline
\end{ltabulary}
\captionof{table}{Controlli della classe RA} 
Il processo di analisi del rischio è per definizione procedurale. Tuttavia, in ambito IT, è spesso supportato da strumenti automatici e modelli per la raccolta di informazioni utili a supporto dell'analisi.
Molte funzionalità di \textit{risk analysis} possono essere quindi automatizzate: tra queste la gestione e gli aggiornamenti di eventuali definizioni dei \textit{tool} utilizzati e il \textit{vulnerability scanning}. 
Per individuare le vulnerabilità possiamo considerare due approcci:
\begin{itemize}
    \item \textbf{Vulnerability scan statico}, che analizza le vulnerabilità sulla base della versione dei prodotti installati, confrontando i numeri di rilascio con le vulnerabilità note.
    \item \textbf{Vulnerability scan dinamico}, che effettua una vera e propria scansione verso il sistema \textit{target}, in modo più invasivo.
\end{itemize}
Il metodo automatico, in questo caso, può essere applicato nel 60\% dei casi. 
