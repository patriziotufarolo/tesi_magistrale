\documentclass[../main.tex]{subfiles}
\begin{document}
\chapter{Moon Cloud: un framework per il monitoraggio e la security assurance}
\section{Introduzione}
In questo capitolo verrà approfondito Moon Cloud\footnote{MOnitoring and assurance ON Cloud, \textit{https://www.moon-cloud.eu/}}, un framework per il monitoraggio e l'assurance di sistemi tradizionali, cloud e Internet of Things, sviluppato dal laboratorio SESAR Lab\footnote{SEcure Service-oriented Architectures Research Lab - \textit{http://sesar.di.unimi.it}} dell'Università degli Studi di Milano.

L'obiettivo del progetto Moon Cloud è quello di implementare una metodologia automatica per la valutazione della sicurezza, delle performance e di altre proprietà non funzionali, offrendo un processo di assurance basato su attività di auditing e raccolta di evidenze.

Le caratteristiche di Moon Cloud sono le seguenti\cite{MoonCloudWebsite}:
\begin{itemize}
    \item \textbf{Framework automatico e personalizzabile} per la raccolta di evidenze, basato su modelli 
    \item \textbf{Copertura di tutto lo stack cloud}, ai livelli \textit{IaaS}, \textit{PaaS}, \textit{SaaS}
    \item \textbf{Possibilità di integrazione con tecnologie pre-esistenti e di terze parti}
\end{itemize}

L'orchestrazione avviene tramite una  \textit{dashboard} grafica erogata in modalità \textit{Software as-a-Service}, la quale si interfaccia con diversi componenti che ne implementano la logica di funzionamento, l'esecuzione dei test, il collezionamento dei risultati e il reporting dello stato di compliance del sistema analizzato.

Moon Cloud nasce sulla scia del progetto europeo FP7 CUMULUS - il cui obiettivo è quello di fornire un framework per la certificazione di servizi cloud mediante l'assessment di proprietà non funzionali\cite{CumulusBigDoc} - da cui riprende alcune caratteristiche architetturali. 

L'obiettivo di questo capitolo è quello di dare al lettore il \texit{know-how} necessario a comprendere le motivazioni di alcune decisioni prese nella realizzazione della tesi, in particolar modo per gestire alcune limitazioni del framework Moon Cloud. 
Di seguito sarà proposta una breve panoramica sulla terminologia utilizzata, sulle componenti principali,  e sui principi di funzionamento dei processi di \textit{assessment} e \textit{compliance} all'interno della piattaforma.


\section{Terminologia}
\begin{itemize}
    \item \textbf{Metriche}: insieme di proprietà non funzionali di cui si vogliono ottenere \textbf{misurazioni}.
    \item \textbf{Controllo}: rappresenta la modalità di raccolta ed elaborazione delle \textit{misurazioni} di una \textit{metrica}.
        Esso è descritto tramite un documento \textit{JSON}\footnote{JSON, JavaScript Object Notation, \textit{http://www.json.org/}}, i cui attributi sono
        \begin{itemize}
            \item \textbf{name}, nome del controllo
            \item \textbf{description}, descrizione del controllo
            \item \textbf{category}, categoria di appartenenza
            \item \textbf{driver-name}, nome del driver che ne implementa il flusso di esecuzione
            \item \textbf{inputs}, dati ricevuti in input
            \item \textbf{outputs}, dati attesi in output (\textit{misurazioni})
        \end{itemize}
    \item \textbf{Driver}: la porzione di codice che implementa il controllo 
    \item \textbf{Test}: istanza di un controllo, che ne rappresenta l'esecuzione con gli input effettivi 
    \item \textbf{Regola di valutazione astratta}, o \textit{AER (abstract evaluation rule)}, regola logica che implementa un processo di compliance. È data dall'aggregazione di controlli mediante operatori \textit{booleani}. I suoi termini possono essere \textit{controlli} e altre \textit{AER}.
    \item \textbf{Regola di valutazione concreta}, o \textit{ER (evaluation rule)}: istanza di una \textit{AER}. I suoi termini possono essere \textit{test} (mappati sui rispettivi \textit{controlli} o altre \textit{ER}.
\end{itemize}


\section{Architettura e componenti}


\newpage
.\\\newpage
.\\\newpage
.\\\newpage
.\\\newpage
\section{Moon Cloud come strumento di verifica della compliance}
\newpage
.\\\newpage
\subsection{Controlli di sicurezza}
\newpage
.\\\newpage
.\\\newpage
.\\\newpage
.\\\newpage
\subsection{Regole di valutazione}
\newpage
.\\\newpage
.\\\newpage
.\\\newpage
.\\\newpage
\end{document}
